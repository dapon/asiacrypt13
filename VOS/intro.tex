\section{Introduction}
\label{sec:intro}
Cloud computing allows users and organizations to outsource
both their {\it data} and {\it computation} to cloud servers.
Imagine that a client $C$ outsources
a database $DB$ (e.g., a SQL database, a key-value store,
a graph, etc.)\@
to an untrusted cloud server $S$.
The client will then make a series of queries to the server.
For example, the client can ask the server to
compute a function over the outsourced $DB$, and return the answer;
the client can also make update queries such as inserting, deleting,
or modifying entries in the database.
Since the cloud server is outside the
client's control, a big challenge is how to guarantee the
{\it privacy} of outsourced data, and the {\it integrity}
of computation performed by the server.

One way to address this problem is to rely on Non-Interactive
Verified Computation (NIVC), first proposed by
Gennaro, Gentry, and Parno~\cite{ggp} (henceforth referred
to as the GGP construction), and
improved by subsequent works~\cite{ggp,GLR11,chung-outsource,DBLP:conf/icalp/ApplebaumIK10,memory-delegation,ParnoRV12,spanprogram,snarglinearproof}.
At a high-level, NIVC schemes allow a client to
outsource a database to a server
%such that (1)~the server gains no information about the
%database \jnote{is this true? [GGP] hides the input $x$ but not the function~$f$, and in this
%case the database defines the function, no?};
%and (2)~the 
such that it can verify the correctness of
the subsequent query results efficiently.
Some schemes additionally guarantee the privacy
of the database, and/or the inputs and outputs to the queries.

There are, however, several drawbacks with most existing schemes:
First, most schemes~\cite{ggp,GLR11,chung-outsource,DBLP:conf/icalp/ApplebaumIK10,memory-delegation,ParnoRV12,spanprogram,snarglinearproof}
require the server to perform
{\it linear} (in the database
size) computation, even when the
query takes {\it sublinear} time to run in the insecure (i.e., unencrypted
and unauthenticated) setting.
This can be a prohibitive overhead in real-world applications.
In fact, most existing database implementations
leverage efficient data structures
such that answering queries (e.g., binary search, range queries)
take sublinear time.

Second,
while insertions, deletions, and updates,
are common-place in practical
applications,
most existing NIVC schemes~\cite{ggp,GLR11,chung-outsource,DBLP:conf/icalp/ApplebaumIK10,ParnoRV12,spanprogram,snarglinearproof}
do not allow efficient updates to the database.
For example, with the GGP construction,
updating the database would require rerunning the preprocessing
algorithm, which takes at least time linear
in the size of the database.


\subsection{Our Main Results and Contributions}
We consider the Random Access Machine (RAM) model of computation, since most real-world databases queries 
can be computed efficiently (i.e., in sublinear time) in the RAM model.
We make the following contributions:

\paragraph{New definitions.}
We formally define the problem of Verifiable Computation
in the RAM model (VC-RAM). We define the full security of VC-RAM using
a simulation-based definition (equivalent to
Universal Composability) assuming an
honest client and a malicious
%\babis{you might want to say curious server as well}
server.
Our formulation naturally supports {\it updates} to the database.
%\jnote{This can be cut --- I don't think it's a big deal to introduce these definitions.}

\ignore{
First, our VC-RAM formulation can be viewed as a generalization
of a line of research
called authenticated data structures\elaine{cite}.
Our definition is also a generalization of what we refer to as
\name, which has
been studied implicitly in several related work \elaine{cite} --
however, still lacks
a complete security definition.
We elaborate on the implications of our result
and additional applications
in Section~\ref{sec:implications}.
}

\paragraph{Constructions with near-optimal
client and server computation.}
We propose two new VC-RAM constructions,
one that is verifiable-only, but does not guarantee privacy;
and a second construction that guarantees verifiability and privacy
simultaneously.


Our constructions achieve {\it near-optimal} computation overhead
for both the client and the server in the following sense.
Let $\lambda$ denote the security parameter,
$n$ denote the memory size, and $|x|$ and $|\res|$ denote
the input and output sizes respectively.
\begin{itemize}
\item
The client's online computation
%\elaine{actually may be $\poly(|x|, \lambda)$, but a bigger
%query can be split into smaller ones to get the following near optimal result}
%\jnote{I don't understand what it means to ``split a query into smaller ones.''}
for each query is $O(|x| + |\res|) \cdot \poly(\lambda)$,
and is independent of $\tau$.
%where $|x|$ and $|\res|$ denote the bit length of the input
%and output respectively, and lambda is the security parameter.
Notice that the client has to pay at least $|x|+|y|$ cost
to read the input and output.
%{\it independent of the time to execute the query};
\item
%the server's computation
%overhead is similar to the plain (i.e., unencrypted and unauthenticated)
%setting.
If a RAM program takes $\tau$ time to compute
in the insecure setting,
our VC-RAM construction requires the server
to perform only $\tau \cdot \poly(\lambda, \log n)$ computation,
%where $\lambda$ is the security parameter, and $n$ is the memory size.
This implies that for queries that take sublinear time,
our server computation is also sublinear.
%\item \babis{what is the size of the proof? I assume $|x|+|y|$? Could we put it here?}
\end{itemize}

\paragraph{Additional implications of our results.}
Our results have several implications beyond VC-RAM,
%We generalize several
%related lines of research, and answer several open questions in these
%areas some of which have eluded the community for years,
%\jnote{Too over-the-top!}
including (1)~{\it authenticated data structures}; (2)~{\it \name};
and (3)~{\it verifiable
``searchable encryption''} in the symmetric-key setting for
{\it general queries}, {\it hiding access patterns}, 
and with {\it sublinear} server computation (for queries that run in
sublinear time).
We now elaborate on the applications and implications of our main
results.

\ignore{
including 1) we generalize a line of research called {\it authenticated
data structures} \elaine{cite};
2) we generalize {\it \name}, which has
been studied implicitly in several related work \elaine{cite} --
however, still lacks
a complete security definition. We also allow
more efficient constructions of \name; and
3) our results also imply {\it non-interactive, verifiable
searchable encryption} in the symmetric-key setting, for
{\it arbitrary queries}, additionally, allowing {\it updates}
and {\it sublinear} server computation for queries that run in
sublinear time.
We now elaborate on the implications and
additional applications of our results
in Section~\ref{sec:implications}.
}


%\subsection{Applications and Implications of Our Main Results}
%\label{sec:implications}

\subsection{Efficient \namebig and Complete Security Definitions}
%\paragraph{More efficient \name and complete security definitions.}
Our first application of VC-RAM is 
%to enable a more efficient
\name (\nameshort). 
%\babis{I would suggest using ARAM (or something that contains the word ``active") for Active oblivious RAM. VOS does not relate at all to the fact that the server is active. So basically, we set out to do ARAM, and we see that in order to achieve it, verifiability is needed. Then, the verifiable part comes for free, it is not that we set out from the beginning to do verifiable oblivious storage}. \jnote{More efficient than what? Who else considered that problem?}
\nameshort is related to, but different
from, oblivious RAM (ORAM)~\cite{oram00}.
%,oram01,oram02,oram03,oram04,oram05,oram06,oram07,oram08,oram09,oram10,oram11,oram12,oram13,oram14}.
Historically, in the context of ORAM the honest server only reads/writes blocks of data as instructed
by the client, but does no computation.
More recently~\cite{LO12,oram13}, researchers have considered a more general model
in which the server performs local computation as well. We refer to 
the latter model as \name.
%\jnote{Could be written better.}
%\jnote{Another point is being lost here. Usually ORAM just considers privacy, i.e., an honest-but-curious
%server. Here we are adding (explicitly) verifiability, i.e., protection against an actively malicious
%server.} \elaine{i dont think so, they consider malicious servers
%that can corrupt data, and use MAC to secure against it.}

%Later, ORAM
%is applied to the cloud computing setting
%to allow a client to store sensitive data with an untrusted
%server, while hiding both the data and access patterns
%from the server.
%In this setting, it makes sense to consider an {\it active}
%server {\it capable of computation}.
%To differentiate, we refer to this problem
%as \name (\nameshort).
%The community has proposed schemes that leverage server
%computation to reduce the round complexity of \nameshort~\cite{LO12,oram13}.

We observe that the \nameshort setting demands a
different security definition as in the original ORAM setting
considered by Goldreich and Ostrovsky~\cite{oram00} -- particularly,
if the server is required to perform computation as part
of the protocol, our definition aims to achieve {\it verifiability}
even when a malicious server 
may arbitrarily deviate from the prescribed
behavior (other than simply corrupting 
the data~\cite{oram00}).

Our security definition for VC-RAM immediately implies
a full-fledged security definition for \nameshort, since
\nameshort is a special case of VC-RAM, in the sense that
the RAM program considered simply reads or writes data.

We construct more efficient
%\babis{here we are comparing ARAM with ORAM, so we cannot really compare. just say that ARAM achieves same security properties with ORAM, yet with better complexities.}
\nameshort schemes in comparison with known
constructions in the ORAM model~\cite{oram00,oram01,oram03,oram07,oram09}, and 
known schemes in the \nameshort model~\cite{oram13, LO12}.
Relying on Fully Homomorphic Encryption
and our verifiable-only VC-RAM construction,
assuming constant client-side storage and a reasonably large block size,
we can construct
a \nameshort scheme with {\it $O(\log n /\log \log n)$
bandwidth overhead}
with $\poly (\log n, \lambda)$ server computation;
alternatively, we can achieve $O(1)$ bandwidth cost
with $O(n^\alpha \log n)\poly(\lambda)$ server computation (where $\alpha <1$ is a constant).
%\elaine{is the dependence on lambda poly for these FHE schemes?,
%i assume it is not just O(lambda)? the latter would be better though.}
%\babis{could you add that our scheme has only one round of interaction?}
Our schemes are
secure with respect to a {\it malicious} server.
In comparison, the best known result in the
constant client storage setting requires $O(\log ^2 n/\log \log n)$
overhead~\cite{oram03} and works in the ORAM model.

Note that this result does NOT violate the known
super-logarithmic lower bound
for ORAM~\cite{oramlower}, since
this lower bound only applies to
the traditional ORAM setting, not to \nameshort.


\subsection{Other Applications and Implications of Our Main Result}
\paragraph{Verifiable and oblivious ``searchable encryption'' in the symmetric-key setting.}
Searchable encryption~\cite{songsearch,DBLP:conf/ccs/CurtmolaGKO06,pkse,BW07,symenc,rangequery,GKPVZ12,DBLP:conf/eurocrypt/LewkoOSTW10,KSW08} allows a client
to outsource a set of encrypted documents to a server,
and later make search queries over the encrypted data.
Searchable encryption %in the symmetric-key setting
has been studied first for keyword queries~\cite{songsearch,DBLP:conf/ccs/CurtmolaGKO06,pkse,dynamicse},
and later for
conjunctive queries~\cite{BW07,rangequery},
inner product queries~\cite{KSW08,DBLP:conf/eurocrypt/LewkoOSTW10}, and general queries~\cite{GKPVZ12}.
Particularly, private-index
functional encryption~\cite{GKPVZ12} implies a searchable encryption
scheme for general queries.

Previous verifiable searchable encryption schemes
suffer from the following drawbacks:
{\it i)} Existing schemes leak access patterns,
i.e., the server can learn which encrypted documents
match the query. Such access pattern leakage has been shown to be
harmful due to statistical attacks~\cite{accesspatternleak}.
{\it ii)} Existing schemes (except schemes for keyword queries~\cite{DBLP:conf/ccs/CurtmolaGKO06,dynamicse})
require linear server computation, namely,
the server needs to perform computation
for every encrypted document in the dataset.
{\it iii)} 
Most existing searchable encryption schemes (with the exception of
Kurosawa \etal~\cite{DBLP:conf/fc/KurosawaO12})
assume a semi-honest server,
and do not
provide verifiability against malicious servers.
For example, a malicious server can potentially leave out
encrypted documents that match the query, violating completeness.
%\babis{actually no, there is one paper on that by Kurosawa~\cite{DBLP:conf/fc/KurosawaO12}, I added it in the refs.bib}.

The task that searchable encryption aims to achieve\footnote{
Technically, VC-RAM is not a searchable encryption scheme,
but a different formulation which better captures
the goal that searchable encryption tried to achieve.
Searchable encryption
is formulated the same way as private-index functional encryption,
i.e., each token can be used to evaluate on {\it any} ciphertext.
}
can be regarded as a special
case of VC-RAM where the queries are searches.
Our construction therefore addresses all of the above drawbacks
by providing a (non-interactive)
{\it verifiable} searchable encryption scheme
for {\it general queries}
in the symmetric-key setting,
allowing {\it sublinear} server computation for sublinear-time
queries, and {\it hiding access patterns}.
%Our scheme is also non-interactive -- notice that
%an interactive scheme exists as a naive application of oblivious RAM.   
%\babis{I would say that VC-RAM with privacy gives a non-interactive efficient  searchable encryption scheme with no leakage at all. Note that such a scheme with interaction does exist by naive application of ORAM.}
\ignore{Our result implies a (non-interactive)
{\it verifiable} searchable encryption scheme for {\it general queries}
in the symmetric-key setting, secure against a malicious server.
Furthermore, unlike existing schemes, we allow efficient index
structures to be built over the dataset,
such that the server perform  {\it sublinear} computation when the
query can  be executed in sublinear time in the insecure setting.
Our scheme also additionally {\it hides access patterns}.
}

\ignore{
\paragraph{Private-index functional encryption
for RAM programs.}
\elaine{it only works in symmetric-key setting! needs to change.}
Somewhat related to the above,
our result also implies a private-index functional encryption
scheme for RAM programs, where one can ``encrypt'' contents of
memory, and later submit ``tokens'' corresponding to a RAM
program.
Given the a token for a RAM program $f$ and encrypted
data $D$, one can evaluate $f(D)$, but learn nothing else.
%Private-index functional encryption for RAMs is very similar to
%a VC-RAM schem where the encrypted data
%need not be updated --
Achieving this requires a
small modification of our main construction,
since the final garbled circuit needs to be modified to release
the decrypted $f(D)$.
We elaborate on this in Appendix~\ref{sec:fe}.

\elaine{actually write this section}
}

\paragraph{Generalization of (private) authenticated data structures.}
%Our work naturally
%unites two lines of research, verified computation~\cite{ggp,GLR11,chung-outsource,DBLP:conf/icalp/ApplebaumIK10,ParnoRV12,spanprogram,snarglinearproof}
%and authenticated data structures~\cite{nn-crcu-98,GoodrichTS01,cpap-rt-07,ptt-oadsmf-10,ptt-aht-08,sets-auth,dgkmns-faxd-04,graph-auth,DBLP:journals/algorithmica/MartelNDGKS04}.
Authenticated data structures~\cite{nn-crcu-98,GoodrichTS01,cpap-rt-07,ptt-oadsmf-10,ptt-aht-08,sets-auth,dgkmns-faxd-04,graph-auth,DBLP:journals/algorithmica/MartelNDGKS04}
allow a user to outsource a dataset to a server while
retaining only a small digest.,
Later the client can
make queries or updates to the dataset,
and additionally verify that
the answers provided by the server are correct.

Authenticated data structures focus on designing
efficient schemes for specific queries such as
(non)-membership queries~\cite{nn-crcu-98,GoodrichTS01},
range queries~\cite{DBLP:journals/algorithmica/MartelNDGKS04}, set operations~\cite{sets-auth},
graph queries~\cite{graph-auth}, etc.
Unlike in the verifiable computation line of research,
authenticated data structures allow the server to compute
in sublinear time for sublinear-time queries.
Existing authenticated data structures
do not take privacy into account in their design.
Even without the privacy consideration,
how to construct efficient (i.e., sublinear server computation) authenticated data structures
for arbitrary databases and queries
remains an open question.

Our work generalizes authenticated data structures and answers
this open question in the affirmative: 1) we support
arbitrary queries and data structures;
2) like in existing
authenticated data structure schemes, the server's
computation is close to the insecure setting;
3) our client's verification cost is independent of
the query's execution time;
4) we additionally offer privacy guarantees
in the two-party setting (but not the three-party setting) --
however, our verifiable-only scheme works both the two-party
and three-party settings.
%and 5) both VC-RAM and authenticated data structures consider
%the non-interactive setting.


%\babis{you can view the three party setting as a two-party setting with many clients. One client only updates and all the other client can query. Since we are using FHE, I am not sure we can support many clients in the two party setting. On the other hand, VC-RAM without privacy can be used to implement any three-party authenticated data structure.}



\ignore{
\elaine{below is note to myself.}
describe database outsourcing scenario

why ram programs: 1) want sublinear server computation overhead;
2) naturally supports updates


first to formalize and define vc-ram.
formulation naturally supports updates.


show provably secure constructions in both the
verifiable-only and fully secure
(i.e., private and verifiable) settings


generalization of authenticated data structure, for arbitrary queries,
and with privacy.
our framework unifies verified computation and
authenticated data structures. the community previously
approached these from different angles and application settings.

applications:

our verifiable-only scheme can be used to build
an \name scheme with $O(\log n)$ bandwidth cost
and constant client state.
Particularly, the idea is to rely on FHE -- but use
verifiable-only VC-RAM to enforce honest server behavior.
\name is basically ORAM with server-side computation.
while server-side computation has been considered previously in [fill]
oram constructions to
improve round complexity, previously adopted
security definitions are
incomplete particularly in the presence of a malicious server.
We address this problem by providing a full-fledged simulation-based
security definition for \name.
This definition is implicit given the full security definition
of VC-RAM, since \name can be considered as a special case of
VCRAM, when the RAM program simply reads or writes memory.

private, verifiable, and non-interactive searches on databases,
supporting updates.

}

%\subsection{Technical Highlights}

\subsection{Other Related Work}
Related work on Non-Interactive Verified Computation~\cite{ggp,GLR11,chung-outsource,DBLP:conf/icalp/ApplebaumIK10,memory-delegation,ParnoRV12,spanprogram,snarglinearproof} and authenticated data structures~\cite{nn-crcu-98,GoodrichTS01,cpap-rt-07,ptt-oadsmf-10,ptt-aht-08,sets-auth,dgkmns-faxd-04,graph-auth,DBLP:journals/algorithmica/MartelNDGKS04}
has been mentioned earlier Section~\ref{sec:intro}.

Garbled RAM, recently proposed by Lu and Ostrovsky~\cite{LO12}
is very closely related to our work.
%can be made to work in the VC-RAM setting -- 
Garbled RAM can be modified to work in our VC-RAM setting --
however, the client's online computation overhead per query
is proportional to the run-time of the RAM program.
Our second construction, the verifiable
%\babis{I think for the verifiable only part, we do not need Garble RAM. Could we make the distinction so that the reader is not confused?} 
and private scheme,
builds on top of the garbled RAM work -- 
for various technical reasons explained later, 
we need a {\it generalized} variant 
of the garbled RAM scheme 
that builds on {\it any} Oblivious RAM compiler, and 
proven secure under 
our new VC-RAM security model  
(a bit stronger than Lu and Ostrovsky's definition~\cite{LO12}).
We refer to this scheme (described in Appendix~\ref{sec:garbledRAM})
as a verifier-inefficient VC-RAM scheme. In 
Section~\ref{sec:VP}, we show how to boost its efficiency
such that the client's online computation is  
independent of the RAM's run-time.

We note that unlike the GGP construction which wraps 
Fully Homomorphic Encryption (FHE) around a garbled circuit,  
wrapping FHE around Garbled RAM does not  
work in our case -- and the reason is similar to 
why the strawman scheme in Section~\ref{sec:VP} fails.

The techniques for our verifiable-only 
VC-RAM scheme are reminiscent of those used by Ben-Sasson 
\etal~\cite{RAMtocircuit}
for reductions from RAMs to delegatable 
succinct constraint satisfaction problems.

Our work is also related to PCPs~\cite{BFLS91,FGL96,pcp,ALM98,Kil92,Kil95}, 
and interactive proofs~\cite{LFKN92,Sha92,babai85,GMR85,GMR89}.
Although we mainly focus on the non-interactive setting.
Our verifiably-only construction relies 
on non-interactive arguments (of knowledge) as 
a building block~\cite{Mic94,Mic00,FS87,BCCT12,GLR11,DFH12,spanprogram}.

