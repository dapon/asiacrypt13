\documentclass[11pt]{article}

\usepackage{fullpage}
\usepackage{color,xspace}
\usepackage{amssymb,amsmath,amsfonts,amsthm}
\usepackage{enumerate}
\usepackage{graphicx}
\usepackage{multirow}
\usepackage{boxedminipage}
\usepackage{subfigure}
\usepackage{xspace}
\usepackage[sort]{natbib}
\usepackage{hyperref}
\renewcommand{\paragraph}[1]{\medskip\noindent{\bf #1}}

%\newcommand{\elaine}[1]{{\color{red}{[Elaine: #1]}}}
%\newcommand{\daniel}[1]{{\footnotesize[Daniel: #1]}}
%\newcommand{\aish}[1]{{\footnotesize[Aishwarya: #1]}}
%\newcommand{\jnote}[1]{{\color{blue}{[Jon: #1]}}}
\newcommand{\elaine}[1]{}
\newcommand{\jnote}[1]{}

\newcommand{\ignore}[1]{}

\newcommand{\etal}{et al.\xspace}

\theoremstyle{plain}

\newtheorem{thm}{Theorem}
\newtheorem{lemma}[thm]{Lemma}
\newtheorem{protocol}[thm]{Protocol}
\newtheorem{definition}[thm]{Definition}
\newtheorem{cor}[thm]{Corollary}
\newtheorem{remark}{Remark}

\newcommand{\G}{\ensuremath{\mathbb{G}}}
\newcommand{\N}{\ensuremath{\mathbb{N}}}
\newcommand{\Z}{\ensuremath{\mathbb{Z}}}
\newcommand{\Q}{\ensuremath{\mathbb{Q}}}

\newcommand{\compind}{\ensuremath{\stackrel{c}{\approx}}}

\newcommand{\RLWE}{{\sf RLWE}}
\newcommand{\SVP}{{\sf SVP}}

\newcommand{\sFE}{{\sf sFE}}
\newcommand{\PE}{{\sf PE}}
\newcommand{\pp}{{\it pp}}
\newcommand{\msk}{{\sf msk}}
%\newcommand{\algTokenGen}{{\sf TokenGen}}

\newcommand{\regev}{{\sf Regev}}
\newcommand{\regevpriv}{{\sf Regev.PrivateKeyGen}}
\newcommand{\regevpub}{{\sf Regev.PublicKeyGen}}
\newcommand{\regevenc}{{\sf Regev.Enc}}
\newcommand{\regevdec}{{\sf Regev.Dec}}

\newcommand{\enc}{{\sf Enc}}
\newcommand{\dec}{{\sf Dec}}

\newcommand{\eps}{\ensuremath{\epsilon}}
\newcommand\defeq{\ensuremath{\stackrel{\text{def}}{=}}}
\newcommand\squareforqed{\hbox{$\blacksquare$}}

\newcommand{\header}[1]{\noindent\textbf{#1}}

\hyphenation{non-empty Mu-thu-krish-nan}

\newcommand{\Read}{\mathsf{read}}
\newcommand{\Write}{\mathsf{write}}
\newcommand{\algAccess}{\mathsf{Access}}


\newcommand{\poly}{{\rm poly}}

\newcommand{\var}{\mathsf{var}}
\newcommand{\rd}{\mathsf{read}}
\newcommand{\wt}{\mathsf{write}}
\newcommand{\st}{\mathsf{cpustate}}
\newcommand{\state}{\mathit{state}}
%\newcommand{\cst}{\mathit{st}}
\newcommand{\cst}{z}
\newcommand{\sst}{Z}
\newcommand{\nextins}{\textsc{NextIns}}
%\newcommand{\inp}{{\sf in}}
\newcommand{\inp}{x}
%\newcommand{\res}{{\sf out}}
\newcommand{\res}{y}


\newcommand{\IdealInit}{\mathsf{ideal\_init}}
\newcommand{\IdealRead}{\mathsf{ideal\_read}}
\newcommand{\IdealWrite}{\mathsf{ideal\_write}}

\newcommand{\rr}{{\sf ReadAndRemove}\xspace}
\newcommand{\add}{{\sf Add}\xspace}
\newcommand{\evict}{{\sf PathEvict}}
\newcommand{\pop}{\textsf{Pop }}
\newcommand{\Pop}{\textsf{Pop}}
\newcommand{\data}{\mathsf{data}}
\newcommand{\blockid}{\textsf{u}}
\newcommand{\op}{\textsf{op}}
\newcommand{\ops}{\textsf{ops}}
\newcommand{\argu}{\textsf{arg}}
\newcommand{\ind}{\textsf{index}}
\newcommand{\erate}{\nu}
\newcommand{\leaf}{\ell}

\newcommand{\found}{\mathit{found}}


\newcommand{\U}{\mathcal{U}}
\newcommand{\algS}{\mathcal{S}}
\newcommand{\algC}{\mathcal{C}}
\newcommand{\algSbar}{\overline{\mathcal{S}}}
\newcommand{\algCbar}{\overline{\mathcal{C}}}
\newcommand{\T}{\mathcal{F}}
\newcommand{\algA}{\mathcal{A}}

\newcommand{\fmpk}{\mathsf{fmpk}}
\newcommand{\fmsk}{\mathsf{fmsk}}
\newcommand{\fsk}{\mathsf{fsk}}
%\newcommand{\tk}{\mathsf{TK}}
\newcommand{\setup}{{\sf Setup}}
\newcommand{\prove}{{\sf Prove}}
\newcommand{\vrfy}{{\sf Vrfy}}


\newcommand{\SNARG}{\mathsf{SNARG}}
\newcommand{\SNARK}{\mathsf{SNARK}}
\newcommand{\tr}{\mathsf{tr}}
\newcommand{\memcheck}{\mathsf{MC}}
\newcommand{\MC}{\mathsf{MC}}
\newcommand{\Gen}{\mathsf{Gen}}
\renewcommand{\P}{\mathsf{Prove}}
\newcommand{\Ex}{\mathsf{Extract}}
\newcommand{\V}{\mathsf{Verify}}
\newcommand{\algSetup}{\mathsf{Setup}}
\newcommand{\algInit}{\mathsf{Init}}
\newcommand{\algKeygen}{\mathsf{KeyGen}}
\newcommand{\algVerify}{\mathsf{Verify}}
\newcommand{\algCheck}{\mathsf{CheckDigest}}
\newcommand{\algCompute}{\mathsf{Compute}}
\newcommand{\algPrepare}{\mathsf{ProbGen}}
\newcommand{\algProbgen}{\mathsf{ProbGen}}
\newcommand{\digest}{d}
\newcommand{\gsk}{\mathsf{gsk}}
\newcommand{\digestn}{\widetilde{d}}
\newcommand{\sk}{\mathit{sk}}
\newcommand{\aux}{\mathsf{aux}}
\newcommand{\ek}{\mathsf{ek}}
\newcommand{\pk}{\mathit{pk}}
\newcommand{\SIM}{\mathsf{Sim}}
\newcommand{\PRF}{\mathsf{PRF}}
\newcommand{\PRP}{\mathsf{PRP}}
\newcommand{\hybrid}{\mathsf{H}}
\newcommand{\vcram}{\mathsf{VCRAM}}
\newcommand{\VCRAM}{\Pi}
\newcommand{\digestfunc}{\mathsf{digest}}
\newcommand{\params}{\mathit{params}}

\newcommand{\MAC}{\mathsf{MAC}}
\newcommand{\GC}{\textsc{GC}}
\newcommand{\RGC}{\textsc{RGC}}
\newcommand{\GRAM}{\textsc{GRAM}}
\newcommand{\Gb}{\mathsf{Garble}}
\newcommand{\Gdc}{{\mathsf{G_D}}}
\newcommand{\GT}{{\mathsf{GarbleTrans}}}
\newcommand{\GI}{\mathsf{GEnc}}
\newcommand{\GDec}{\mathsf{GDec}}
\newcommand{\GSetup}{\mathsf{GSetup}}
\newcommand{\Tr}{\textsc{Trans}}
\newcommand{\mTr}{\text{m}\textsc{Trans}}
\newcommand{\cTr}{\text{c}\textsc{Trans}}
\newcommand{\GE}{\mathsf{GEval}}
\newcommand{\GTE}{\mathsf{GTEval}}
\newcommand{\ORAM}{\mathsf{ORAM}}

\newcommand{\pCompile}{\mathsf{pCompile}}
\newcommand{\oCompile}{\mathsf{oCompile}}
\newcommand{\RAM}{f}


\newcommand{\Dec}{\textsf{Dec}}


\newcommand{\bool}{\{0,1\}}

\newcommand{\find}{\mathsf{find}}
\newcommand{\fhe}{\mathsf{FHE}}
\newcommand{\kg}{\textsf{KeyGen}}
%\newcommand{\enc}{\textsf{enc}}
%\newcommand{\dec}{\textsf{dec}}
\newcommand{\eval}{\textsf{Eval}}
\newcommand{\cnt}{\mathit{cnt}}
\newcommand{\addr}{\mathsf{addr}}
\newcommand{\raddr}{\mathsf{raddr}}
\newcommand{\waddr}{\mathsf{waddr}}
\newcommand{\laddr}{\mathsf{laddr}}
\newcommand{\tg}{\mathsf{tag}}
\newcommand{\itg}{\mathsf{itag}}
\newcommand{\itgs}{\mathsf{itags}}
\newcommand{\otg}{\mathsf{otag}}
\newcommand{\otgs}{\mathsf{otags}}
\newcommand{\tgs}{\mathsf{tags}}
\newcommand{\symb}{\mathsf{symbol}}

\newcommand{\NN}{\mathbb{N}}
\newcommand{\C}{\mathcal{C}}
\newcommand{\A}{\mathcal{A}}
\newcommand{\env}{\mathcal{Z}}

%\newcommand{\advantage}{\mathbf{Adv}}
%\newcommand{\cpa}{\sf CPA}
\newcommand{\he}{\mathcal{HE}}
\newcommand{\prob}{\ensuremath{\mathbf{Pr}}}
%\def\sk{\mathit{sk}}
%\def\pk{\mathit{pk}}

\newcommand{\name}{verifiable oblivious storage\xspace}
\newcommand{\namebig}{Verifiable Oblivious Storage\xspace}
\newcommand{\nameshort}{VOS\xspace}


\begin{document}

\title{{\bf Non-Interactive Verifiable RAM Computation}}
%\author{{\small Anonymous Submission}}
\date{\vspace{-6mm}}
%\author{
%Daniel Apon\thanks{University of Maryland. Email: \{dapon,jkatz,elaine,aish\}@cs.umd.edu}
%\and Jonathan Katz$^*$
%\and Charalampos Papamanthou\thanks{University of California, Berkeley. Email: cpap@cs.berkeley.edu}
%\and Elaine Shi$^*$
%\and Aishwarya Thiruvengadam$^*$
%}

\maketitle

\begin{abstract}
Consider a client who wishes to outsource a large database to an
untrusted server, and subsequently would like to make queries
and updates to the database. The client wishes to guarantee
both privacy and verifiability.
Most existing %non-interactive \babis{interactive too}
verifiable computation
fail to address this problem satisfactorily, due to the fact
that the server's computation per query
is linear in the size of the database.

We propose non-interactive verifiable RAM computation -- since most
real-world database queries (e.g., range queries, binary search,
graph queries) can be implemented efficiently in the RAM model.
We propose two constructions: one that guarantees verifiability but
does not offer privacy; and a second one that guarantees both.
Our constructions have near-optimal efficiency: the server's computation
per query is not significantly more than the run-time of the
RAM program; the client's online computation per query is independent
of the run-time of the RAM, or the database size.

We show that our VC-RAM model is powerful, and give several applications.
Particularly, we show how to rely on verifiable-only VC-RAM
construction to build
\name schemes (i.e., oblivious
RAM with active server computation)
that consume asymptotically less bandwidth overhead
than existing oblivious RAM schemes, while ensuring
verifiability against an arbitrarily malicious server.
\end{abstract}

\noindent {\small {\bf Keywords}: verifiable RAM computation; verifiable oblivious storage;
oblivious RAM
}

\section{Introduction}
\label{sec:intro}
Cloud computing allows users and organizations to outsource
both their {\it data} and {\it computation} to cloud servers.
Imagine that a client $C$ outsources
a database $DB$ (e.g., a SQL database, a key-value store,
a graph, etc.)\@
to an untrusted cloud server $S$.
The client will then make a series of queries to the server.
For example, the client can ask the server to
compute a function over the outsourced $DB$, and return the answer;
the client can also make update queries such as inserting, deleting,
or modifying entries in the database.
Since the cloud server is outside the
client's control, a big challenge is how to guarantee the
{\it privacy} of outsourced data, and the {\it integrity}
of computation performed by the server.

One way to address this problem is to rely on Non-Interactive
Verified Computation (NIVC), first proposed by
Gennaro, Gentry, and Parno~\cite{ggp} (henceforth referred
to as the GGP construction), and
improved by subsequent works~\cite{ggp,GLR11,chung-outsource,DBLP:conf/icalp/ApplebaumIK10,memory-delegation,ParnoRV12,spanprogram,snarglinearproof}.
At a high-level, NIVC schemes allow a client to
outsource a database to a server
%such that (1)~the server gains no information about the
%database \jnote{is this true? [GGP] hides the input $x$ but not the function~$f$, and in this
%case the database defines the function, no?};
%and (2)~the 
such that it can verify the correctness of
the subsequent query results efficiently.
Some schemes additionally guarantee the privacy
of the database, and/or the inputs and outputs to the queries.

There are, however, several drawbacks with most existing schemes:
First, most schemes~\cite{ggp,GLR11,chung-outsource,DBLP:conf/icalp/ApplebaumIK10,memory-delegation,ParnoRV12,spanprogram,snarglinearproof}
require the server to perform
{\it linear} (in the database
size) computation, even when the
query takes {\it sublinear} time to run in the insecure (i.e., unencrypted
and unauthenticated) setting.
This can be a prohibitive overhead in real-world applications.
In fact, most existing database implementations
leverage efficient data structures
such that answering queries (e.g., binary search, range queries)
take sublinear time.

Second,
while insertions, deletions, and updates,
are common-place in practical
applications,
most existing NIVC schemes~\cite{ggp,GLR11,chung-outsource,DBLP:conf/icalp/ApplebaumIK10,ParnoRV12,spanprogram,snarglinearproof}
do not allow efficient updates to the database.
For example, with the GGP construction,
updating the database would require rerunning the preprocessing
algorithm, which takes at least time linear
in the size of the database.


\subsection{Our Main Results and Contributions}
We consider the Random Access Machine (RAM) model of computation, since most real-world databases queries 
can be computed efficiently (i.e., in sublinear time) in the RAM model.
We make the following contributions:

\paragraph{New definitions.}
We formally define the problem of Verifiable Computation
in the RAM model (VC-RAM). We define the full security of VC-RAM using
a simulation-based definition (equivalent to
Universal Composability) assuming an
honest client and a malicious
%\babis{you might want to say curious server as well}
server.
Our formulation naturally supports {\it updates} to the database.
%\jnote{This can be cut --- I don't think it's a big deal to introduce these definitions.}

\ignore{
First, our VC-RAM formulation can be viewed as a generalization
of a line of research
called authenticated data structures\elaine{cite}.
Our definition is also a generalization of what we refer to as
\name, which has
been studied implicitly in several related work \elaine{cite} --
however, still lacks
a complete security definition.
We elaborate on the implications of our result
and additional applications
in Section~\ref{sec:implications}.
}

\paragraph{Constructions with near-optimal
client and server computation.}
We propose two new VC-RAM constructions,
one that is verifiable-only, but does not guarantee privacy;
and a second construction that guarantees verifiability and privacy
simultaneously.


Our constructions achieve {\it near-optimal} computation overhead
for both the client and the server in the following sense.
Let $\lambda$ denote the security parameter,
$n$ denote the memory size, and $|x|$ and $|\res|$ denote
the input and output sizes respectively.
\begin{itemize}
\item
The client's online computation
%\elaine{actually may be $\poly(|x|, \lambda)$, but a bigger
%query can be split into smaller ones to get the following near optimal result}
%\jnote{I don't understand what it means to ``split a query into smaller ones.''}
for each query is $O(|x| + |\res|) \cdot \poly(\lambda)$,
and is independent of $\tau$.
%where $|x|$ and $|\res|$ denote the bit length of the input
%and output respectively, and lambda is the security parameter.
Notice that the client has to pay at least $|x|+|y|$ cost
to read the input and output.
%{\it independent of the time to execute the query};
\item
%the server's computation
%overhead is similar to the plain (i.e., unencrypted and unauthenticated)
%setting.
If a RAM program takes $\tau$ time to compute
in the insecure setting,
our VC-RAM construction requires the server
to perform only $\tau \cdot \poly(\lambda, \log n)$ computation,
%where $\lambda$ is the security parameter, and $n$ is the memory size.
This implies that for queries that take sublinear time,
our server computation is also sublinear.
%\item \babis{what is the size of the proof? I assume $|x|+|y|$? Could we put it here?}
\end{itemize}

\paragraph{Additional implications of our results.}
Our results have several implications beyond VC-RAM,
%We generalize several
%related lines of research, and answer several open questions in these
%areas some of which have eluded the community for years,
%\jnote{Too over-the-top!}
including (1)~{\it authenticated data structures}; (2)~{\it \name};
and (3)~{\it verifiable
``searchable encryption''} in the symmetric-key setting for
{\it general queries}, {\it hiding access patterns}, 
and with {\it sublinear} server computation (for queries that run in
sublinear time).
We now elaborate on the applications and implications of our main
results.

\ignore{
including 1) we generalize a line of research called {\it authenticated
data structures} \elaine{cite};
2) we generalize {\it \name}, which has
been studied implicitly in several related work \elaine{cite} --
however, still lacks
a complete security definition. We also allow
more efficient constructions of \name; and
3) our results also imply {\it non-interactive, verifiable
searchable encryption} in the symmetric-key setting, for
{\it arbitrary queries}, additionally, allowing {\it updates}
and {\it sublinear} server computation for queries that run in
sublinear time.
We now elaborate on the implications and
additional applications of our results
in Section~\ref{sec:implications}.
}


%\subsection{Applications and Implications of Our Main Results}
%\label{sec:implications}

\subsection{Efficient \namebig and Complete Security Definitions}
%\paragraph{More efficient \name and complete security definitions.}
Our first application of VC-RAM is 
%to enable a more efficient
\name (\nameshort). 
%\babis{I would suggest using ARAM (or something that contains the word ``active") for Active oblivious RAM. VOS does not relate at all to the fact that the server is active. So basically, we set out to do ARAM, and we see that in order to achieve it, verifiability is needed. Then, the verifiable part comes for free, it is not that we set out from the beginning to do verifiable oblivious storage}. \jnote{More efficient than what? Who else considered that problem?}
\nameshort is related to, but different
from, oblivious RAM (ORAM)~\cite{oram00}.
%,oram01,oram02,oram03,oram04,oram05,oram06,oram07,oram08,oram09,oram10,oram11,oram12,oram13,oram14}.
Historically, in the context of ORAM the honest server only reads/writes blocks of data as instructed
by the client, but does no computation.
More recently~\cite{LO12,oram13}, researchers have considered a more general model
in which the server performs local computation as well. We refer to 
the latter model as \name.
%\jnote{Could be written better.}
%\jnote{Another point is being lost here. Usually ORAM just considers privacy, i.e., an honest-but-curious
%server. Here we are adding (explicitly) verifiability, i.e., protection against an actively malicious
%server.} \elaine{i dont think so, they consider malicious servers
%that can corrupt data, and use MAC to secure against it.}

%Later, ORAM
%is applied to the cloud computing setting
%to allow a client to store sensitive data with an untrusted
%server, while hiding both the data and access patterns
%from the server.
%In this setting, it makes sense to consider an {\it active}
%server {\it capable of computation}.
%To differentiate, we refer to this problem
%as \name (\nameshort).
%The community has proposed schemes that leverage server
%computation to reduce the round complexity of \nameshort~\cite{LO12,oram13}.

We observe that the \nameshort setting demands a
different security definition as in the original ORAM setting
considered by Goldreich and Ostrovsky~\cite{oram00} -- particularly,
if the server is required to perform computation as part
of the protocol, our definition aims to achieve {\it verifiability}
even when a malicious server 
may arbitrarily deviate from the prescribed
behavior (other than simply corrupting 
the data~\cite{oram00}).

Our security definition for VC-RAM immediately implies
a full-fledged security definition for \nameshort, since
\nameshort is a special case of VC-RAM, in the sense that
the RAM program considered simply reads or writes data.

We construct more efficient
%\babis{here we are comparing ARAM with ORAM, so we cannot really compare. just say that ARAM achieves same security properties with ORAM, yet with better complexities.}
\nameshort schemes in comparison with known
constructions in the ORAM model~\cite{oram00,oram01,oram03,oram07,oram09}, and 
known schemes in the \nameshort model~\cite{oram13, LO12}.
Relying on Fully Homomorphic Encryption
and our verifiable-only VC-RAM construction,
assuming constant client-side storage and a reasonably large block size,
we can construct
a \nameshort scheme with {\it $O(\log n /\log \log n)$
bandwidth overhead}
with $\poly (\log n, \lambda)$ server computation;
alternatively, we can achieve $O(1)$ bandwidth cost
with $O(n^\alpha \log n)\poly(\lambda)$ server computation (where $\alpha <1$ is a constant).
%\elaine{is the dependence on lambda poly for these FHE schemes?,
%i assume it is not just O(lambda)? the latter would be better though.}
%\babis{could you add that our scheme has only one round of interaction?}
Our schemes are
secure with respect to a {\it malicious} server.
In comparison, the best known result in the
constant client storage setting requires $O(\log ^2 n/\log \log n)$
overhead~\cite{oram03} and works in the ORAM model.

Note that this result does NOT violate the known
super-logarithmic lower bound
for ORAM~\cite{oramlower}, since
this lower bound only applies to
the traditional ORAM setting, not to \nameshort.


\subsection{Other Applications and Implications of Our Main Result}
\paragraph{Verifiable and oblivious ``searchable encryption'' in the symmetric-key setting.}
Searchable encryption~\cite{songsearch,DBLP:conf/ccs/CurtmolaGKO06,pkse,BW07,symenc,rangequery,GKPVZ12,DBLP:conf/eurocrypt/LewkoOSTW10,KSW08} allows a client
to outsource a set of encrypted documents to a server,
and later make search queries over the encrypted data.
Searchable encryption %in the symmetric-key setting
has been studied first for keyword queries~\cite{songsearch,DBLP:conf/ccs/CurtmolaGKO06,pkse,dynamicse},
and later for
conjunctive queries~\cite{BW07,rangequery},
inner product queries~\cite{KSW08,DBLP:conf/eurocrypt/LewkoOSTW10}, and general queries~\cite{GKPVZ12}.
Particularly, private-index
functional encryption~\cite{GKPVZ12} implies a searchable encryption
scheme for general queries.

Previous verifiable searchable encryption schemes
suffer from the following drawbacks:
{\it i)} Existing schemes leak access patterns,
i.e., the server can learn which encrypted documents
match the query. Such access pattern leakage has been shown to be
harmful due to statistical attacks~\cite{accesspatternleak}.
{\it ii)} Existing schemes (except schemes for keyword queries~\cite{DBLP:conf/ccs/CurtmolaGKO06,dynamicse})
require linear server computation, namely,
the server needs to perform computation
for every encrypted document in the dataset.
{\it iii)} 
Most existing searchable encryption schemes (with the exception of
Kurosawa \etal~\cite{DBLP:conf/fc/KurosawaO12})
assume a semi-honest server,
and do not
provide verifiability against malicious servers.
For example, a malicious server can potentially leave out
encrypted documents that match the query, violating completeness.
%\babis{actually no, there is one paper on that by Kurosawa~\cite{DBLP:conf/fc/KurosawaO12}, I added it in the refs.bib}.

The task that searchable encryption aims to achieve\footnote{
Technically, VC-RAM is not a searchable encryption scheme,
but a different formulation which better captures
the goal that searchable encryption tried to achieve.
Searchable encryption
is formulated the same way as private-index functional encryption,
i.e., each token can be used to evaluate on {\it any} ciphertext.
}
can be regarded as a special
case of VC-RAM where the queries are searches.
Our construction therefore addresses all of the above drawbacks
by providing a (non-interactive)
{\it verifiable} searchable encryption scheme
for {\it general queries}
in the symmetric-key setting,
allowing {\it sublinear} server computation for sublinear-time
queries, and {\it hiding access patterns}.
%Our scheme is also non-interactive -- notice that
%an interactive scheme exists as a naive application of oblivious RAM.   
%\babis{I would say that VC-RAM with privacy gives a non-interactive efficient  searchable encryption scheme with no leakage at all. Note that such a scheme with interaction does exist by naive application of ORAM.}
\ignore{Our result implies a (non-interactive)
{\it verifiable} searchable encryption scheme for {\it general queries}
in the symmetric-key setting, secure against a malicious server.
Furthermore, unlike existing schemes, we allow efficient index
structures to be built over the dataset,
such that the server perform  {\it sublinear} computation when the
query can  be executed in sublinear time in the insecure setting.
Our scheme also additionally {\it hides access patterns}.
}

\ignore{
\paragraph{Private-index functional encryption
for RAM programs.}
\elaine{it only works in symmetric-key setting! needs to change.}
Somewhat related to the above,
our result also implies a private-index functional encryption
scheme for RAM programs, where one can ``encrypt'' contents of
memory, and later submit ``tokens'' corresponding to a RAM
program.
Given the a token for a RAM program $f$ and encrypted
data $D$, one can evaluate $f(D)$, but learn nothing else.
%Private-index functional encryption for RAMs is very similar to
%a VC-RAM schem where the encrypted data
%need not be updated --
Achieving this requires a
small modification of our main construction,
since the final garbled circuit needs to be modified to release
the decrypted $f(D)$.
We elaborate on this in Appendix~\ref{sec:fe}.

\elaine{actually write this section}
}

\paragraph{Generalization of (private) authenticated data structures.}
%Our work naturally
%unites two lines of research, verified computation~\cite{ggp,GLR11,chung-outsource,DBLP:conf/icalp/ApplebaumIK10,ParnoRV12,spanprogram,snarglinearproof}
%and authenticated data structures~\cite{nn-crcu-98,GoodrichTS01,cpap-rt-07,ptt-oadsmf-10,ptt-aht-08,sets-auth,dgkmns-faxd-04,graph-auth,DBLP:journals/algorithmica/MartelNDGKS04}.
Authenticated data structures~\cite{nn-crcu-98,GoodrichTS01,cpap-rt-07,ptt-oadsmf-10,ptt-aht-08,sets-auth,dgkmns-faxd-04,graph-auth,DBLP:journals/algorithmica/MartelNDGKS04}
allow a user to outsource a dataset to a server while
retaining only a small digest.,
Later the client can
make queries or updates to the dataset,
and additionally verify that
the answers provided by the server are correct.

Authenticated data structures focus on designing
efficient schemes for specific queries such as
(non)-membership queries~\cite{nn-crcu-98,GoodrichTS01},
range queries~\cite{DBLP:journals/algorithmica/MartelNDGKS04}, set operations~\cite{sets-auth},
graph queries~\cite{graph-auth}, etc.
Unlike in the verifiable computation line of research,
authenticated data structures allow the server to compute
in sublinear time for sublinear-time queries.
Existing authenticated data structures
do not take privacy into account in their design.
Even without the privacy consideration,
how to construct efficient (i.e., sublinear server computation) authenticated data structures
for arbitrary databases and queries
remains an open question.

Our work generalizes authenticated data structures and answers
this open question in the affirmative: 1) we support
arbitrary queries and data structures;
2) like in existing
authenticated data structure schemes, the server's
computation is close to the insecure setting;
3) our client's verification cost is independent of
the query's execution time;
4) we additionally offer privacy guarantees
in the two-party setting (but not the three-party setting) --
however, our verifiable-only scheme works both the two-party
and three-party settings.
%and 5) both VC-RAM and authenticated data structures consider
%the non-interactive setting.


%\babis{you can view the three party setting as a two-party setting with many clients. One client only updates and all the other client can query. Since we are using FHE, I am not sure we can support many clients in the two party setting. On the other hand, VC-RAM without privacy can be used to implement any three-party authenticated data structure.}



\ignore{
\elaine{below is note to myself.}
describe database outsourcing scenario

why ram programs: 1) want sublinear server computation overhead;
2) naturally supports updates


first to formalize and define vc-ram.
formulation naturally supports updates.


show provably secure constructions in both the
verifiable-only and fully secure
(i.e., private and verifiable) settings


generalization of authenticated data structure, for arbitrary queries,
and with privacy.
our framework unifies verified computation and
authenticated data structures. the community previously
approached these from different angles and application settings.

applications:

our verifiable-only scheme can be used to build
an \name scheme with $O(\log n)$ bandwidth cost
and constant client state.
Particularly, the idea is to rely on FHE -- but use
verifiable-only VC-RAM to enforce honest server behavior.
\name is basically ORAM with server-side computation.
while server-side computation has been considered previously in [fill]
oram constructions to
improve round complexity, previously adopted
security definitions are
incomplete particularly in the presence of a malicious server.
We address this problem by providing a full-fledged simulation-based
security definition for \name.
This definition is implicit given the full security definition
of VC-RAM, since \name can be considered as a special case of
VCRAM, when the RAM program simply reads or writes memory.

private, verifiable, and non-interactive searches on databases,
supporting updates.

}

%\subsection{Technical Highlights}

\subsection{Other Related Work}
Related work on Non-Interactive Verified Computation~\cite{ggp,GLR11,chung-outsource,DBLP:conf/icalp/ApplebaumIK10,memory-delegation,ParnoRV12,spanprogram,snarglinearproof} and authenticated data structures~\cite{nn-crcu-98,GoodrichTS01,cpap-rt-07,ptt-oadsmf-10,ptt-aht-08,sets-auth,dgkmns-faxd-04,graph-auth,DBLP:journals/algorithmica/MartelNDGKS04}
has been mentioned earlier Section~\ref{sec:intro}.

Garbled RAM, recently proposed by Lu and Ostrovsky~\cite{LO12}
is very closely related to our work.
%can be made to work in the VC-RAM setting -- 
Garbled RAM can be modified to work in our VC-RAM setting --
however, the client's online computation overhead per query
is proportional to the run-time of the RAM program.
Our second construction, the verifiable
%\babis{I think for the verifiable only part, we do not need Garble RAM. Could we make the distinction so that the reader is not confused?} 
and private scheme,
builds on top of the garbled RAM work -- 
for various technical reasons explained later, 
we need a {\it generalized} variant 
of the garbled RAM scheme 
that builds on {\it any} Oblivious RAM compiler, and 
proven secure under 
our new VC-RAM security model  
(a bit stronger than Lu and Ostrovsky's definition~\cite{LO12}).
We refer to this scheme (described in Appendix~\ref{sec:garbledRAM})
as a verifier-inefficient VC-RAM scheme. In 
Section~\ref{sec:VP}, we show how to boost its efficiency
such that the client's online computation is  
independent of the RAM's run-time.

We note that unlike the GGP construction which wraps 
Fully Homomorphic Encryption (FHE) around a garbled circuit,  
wrapping FHE around Garbled RAM does not  
work in our case -- and the reason is similar to 
why the strawman scheme in Section~\ref{sec:VP} fails.

The techniques for our verifiable-only 
VC-RAM scheme are reminiscent of those used by Ben-Sasson 
\etal~\cite{RAMtocircuit}
for reductions from RAMs to delegatable 
succinct constraint satisfaction problems.

Our work is also related to PCPs~\cite{BFLS91,FGL96,pcp,ALM98,Kil92,Kil95}, 
and interactive proofs~\cite{LFKN92,Sha92,babai85,GMR85,GMR89}.
Although we mainly focus on the non-interactive setting.
Our verifiably-only construction relies 
on non-interactive arguments (of knowledge) as 
a building block~\cite{Mic94,Mic00,FS87,BCCT12,GLR11,DFH12,spanprogram}.


\section{Preliminaries}
\label{sec:prelim}




\subsection{The Random Access Machine (RAM) Model}
\label{sec:def-RAMs}

\paragraph{RAMs and RAM programs.}
We are interested in modeling computation on a \emph{von~Neumann architecture}.
Here, we have a CPU with random access to some memory~$D$ that stores $n$ words
each of length~$\ell$.
We let $\nextins$ denote the CPU's next-instruction function, which takes as input
some small, local state $\st$ along with the last-read word of memory, and outputs
a write address
$\waddr_t$, a read address $\raddr_t$,
and a value $\data_t$ to be written to~$D[\waddr_t]$.
A RAM \emph{program} $f$ is a sequence of instructions (i.e., executable code)
which is stored in some pre-allocated portion of~$D$.
Other data (e.g., a database, a graph)
may be stored in~$D$ as well before execution begins.
With $D$ initialized as described,
we may then run $f$ on some input~$x$ (also called a \emph{query})
by setting\footnote{This assumes $x$ is ``short.''
If $x$ is ``large,'' we can store it in $D$;
see Section~\ref{sec:VP}.} $\st_0=x$ and then running
\[
\begin{array}{l}
{\sf fetched}_0 = \perp\\
\text{For } t = 1, 2, \ldots:\\
\qquad \left(\data_t, \waddr_{t}, \raddr_{t}, \st_{t}\right) :=
\nextins\left({\sf fetched}_{t-1}, \st_{t-1}\right)\\
\qquad {\sf fetched}_t := D[\raddr_{t}]\\
\qquad D[\waddr_t] := \data_t
\end{array}
\]
\jnote{Do we care about the order of read/write in case $\raddr_t=\waddr_t$? Or do we
assume that never happens?}
We assume that any program $f$ we consider has associated with it a time bound $\tau$
such that the program runs for exactly $\tau$ steps for all queries.
Thus, the execution above ends when $t=\tau$, at which point (by convention)
the final output is~$\data_\tau$.


%A RAM program $f$ is defined to be a RAM description
%along with an initial {\it configuration} $(D, \st)$, where $D[1..n]$
%is the initial memory array (containing code and data), and $\st$ denotes
%the initial CPU states. \jnote{Isn't the initial state always the same?}

\ignore{
We later use the notation ${\sf RAM}: = (D, \st, \nextins, \params)$
to denote a RAM with its initial configuration $D$ and $\st$,
the next instruction circuit $\nextins$, and $\params = (n, \ell, |st|, |x|, |\res|)$,
where $n$ is the size of the memory array, $\ell$ is the bit-length
of each memory word, 
}

\paragraph{Repeated queries.}
We are interested in the case where a RAM program is repeatedly executed a multiple
queries, with the contents of $D$ possibly being updated as the queries are answered.
(E.g., some inputs might represent an update to the underlying database itself, or
might update the data during the course of computing the output.)
If $D$ denotes the initial contents of the memory (including $f$ itself, which we
assume is not being modified), then we write
$\res_m \leftarrow f_D(x_1, x_2, \ldots, x_m)$
to denote that the result of answering the $m$th query in the sequence $x_1, \ldots, x_m$
is~$\res_m$.
\elaine{double-check paper to make sure notation is consistent.}

%\jnote{We may just want to set $\ell=\ell'=O(\log n)$ for
%concreteness and simplicity. Do we ever need $|\inp|$ bigger than $O(\log n)$?}


\paragraph{Assumptions.}
If $\lambda$ denotes the security parameter,
we assume that the number of memory words $n = \poly(\lambda)$,
the total number of queries $Q = \poly(\lambda)$, and the program executes for
$\tau = \poly(\lambda)$ steps.
We also assume that the bit-length of each memory word 
and the CPU states $\ell = O(\log \lambda), |\st| = O(\log \lambda)$.
Since the $\nextins$ function takes in $O(\log \lambda)$ bits and
outputs $O(\log \lambda)$ bits, the $\nextins$ function can be expressed with  
a size-$\tilde{O}(\lambda)$ circuit.
\elaine{do we need to assume the size of the $\nextins$ circuit?}
%We assume that the $\nextins$ circuit computes
%in $\poly \log \lambda$ time.
%and the bit length of each input and output
%$|x| = \Theta(\log n)$, and $|\res| = \Theta(\log n)$.
%In practice, if bigger inputs and outputs
%are required, each query can easily be broken down into multiple
%``smaller'' queries where inputs and outputs are of size $\Theta(\log n)$.
%\elaine{actually may need those to be Theta(lambda) or something for
%the cost analysis}


%In every time step:

%\paragraph{RAM program.} \jnote{This needs to be cleaned up a bit.}
%We use the same formal definition of a RAM program as in
%Gordon \etal~\cite{katzram}.
%Let $n$ denote the memory size.
%We consider a RAM program $f$ that takes in a (small) input $\inp$
%of fixed bit length $\ell'$, and works on a memory
%array $D[1..n]$ of size~$n$, where each word
%has bit length $\ell$. \jnote{We may just want to set $\ell=\ell'=O(\log n)$ for
%concreteness and simplicity. Do we ever need $|\inp|$ bigger than $O(\log n)$?}
\ignore{
The memory array $D$ is accessed using
a sequence of read or write instructions.
Any such instruction
$I \in \left(\{\rd, \wt, \mathsf{stop}\} \times \N \times \{0,1\}^*\right)$
takes the form $(\wt, a, v)$ (i.e., write
the value $v$ to address $a$);
or $(\rd, a, \bot)$ (i.e., read the data stored at address $a$).
We also assume a designated $\mathsf{stop}$ instruction of the form
$(\mathsf{stop}, \res)$  that indicates the termination of the RAM program
with output $\res$. \jnote{We need some notation for executing RAM program $\Pi$ on array $D$
and input~$x$.}

Formally, a RAM program is defined by a
``next instruction'' function $\nextins$ which, given the current CPU state,
and a value $v$ which will always be equal to the last value read from memory,
outputs the next instruction and an updated state.
Assume $D$ has $n$ entries, each $\ell$ bits long, then
we can view execution of a RAM program on inputs $D, \inp$ as follows:

Let $\st := (1^{\log n}, 1^\ell, \mathsf{start}, \inp)$ and $v = 0^\ell$.
Until termination do:
\begin{enumerate}
\item
Compute $(I, \st') \leftarrow \nextins(\st, v)$. Set $\st = \st'$.
\item
If $I = (\mathsf{stop}, \res)$, then terminate with output $\res$.
\item
If $I = (\wt, a, v')$, then set $D[a] = v'$.
\item
If $I = (\rd, a, \bot)$, then set $v = D[a]$.
\end{enumerate}

We require that the size of $\st$, and the space needed to compute $\nextins$, is polynomial in
$\log n$, $\ell$, and $|\inp|$.
\elaine{think about this assumption}
We say that a RAM program $f$ completes in time $\tau$, if
the number of instructions issued in the above execution is exactly $\tau$.


We stress that the memory contents of $D$ may change during the execution of the RAM program.
We use the notation $(D', \res) \leftarrow f(D, \inp)$
to denote the outcome of executing $f$ over data array $D$, and input $\inp$. Specifically,
$f$ outputs an answer $\res$, and may make modifications to the data array $D$, resulting
in $D'$.
Given initial memory array $D_0$ and queries $\inp_1, \inp_2, \ldots,
\inp_m$, we recursively define $(D_i, \res_i) = f(D_{i-1}, \inp_i)$ and call $\res_i$ the
\textit{correct answer}
for the $i$th query. We also write $(D_i, \res_i) = f(D_0, \inp_1, \ldots, \inp_i)$. \jnote{Clean this up.}
}




\subsection{Memory Checking}
\label{app:MC}
We recast memory checking~\cite{blum-memories-94} in a form better suited for our purposes,
however it is clear that our definition is equivalent to that used in prior work.
We use the same notation as in Section~\ref{sec:def-RAMs}. If $D$ is a memory array,
then the ``instruction'' $I=(\data, \waddr, \raddr)$
sets $D[\waddr] = \data$ and
returns
${\sf fetched} = D[\raddr]$.
If a sequence of instructions $I_1=(\data_1, \waddr_1, \raddr_1), \ldots,
I_m = (\data_m, \waddr_m, \raddr_m)$ is executed, the correct answer to the final
instruction (i.e., the final value~${\sf fetched}_m$) is defined in the obvious way:
if $\raddr_m \not \in \{\waddr_1, \ldots, \waddr_{m-1}\}$ then ${\sf fetched}_m = D[\raddr_m]$,
i.e., the contents of the original memory at location~$\raddr_m$. Otherwise, let
$t<m$ be maximal such that $\raddr_m = \waddr_t$; then ${\sf fetched}_m = \data_t$.

\begin{definition}
A {\sf memory-checking scheme} consists of algorithms
$(\setup, \prove, \vrfy)$ such that:
\begin{itemize}
\item $\setup$ takes as input a security parameter~$1^\lambda$ and
an array $D$, and outputs a transformed array $\tilde D$ along with
a digest~$d$.
\item $\prove$ takes as input $\tilde D$ and an instruction $I=(\data, \waddr, \raddr)$.
It outputs an updated array $\tilde D$, a value ${\sf fetched}$, and a proof~$\pi$.
% where $I$ either takes
%the form $I = ({\sf read}, a)$ or
%$I=({\sf write}, a, v)$. Then:
%\begin{itemize}
%\item If $I$ is a read instruction, it outputs a value $v$ and a proof~$\pi$.
%\item If $I$ is a write instruction, it outputs an updated array $\tilde D$, a (new) digest~$d$,
%and a proof~$\pi$.
\item $\vrfy$ takes as input a digest~$d$, an instruction $I=(\data, \waddr, \raddr)$,
a value ${\sf fetched}$, and a proof~$\pi$. It outputs
a bit~$b$ and an updated digest~$d$.
\end{itemize}
The correctness requirement is that for any initial array~$D$,
and any (adaptively chosen) sequence of instructions $I_1, \ldots, I_m$, if we run
\[(\tilde D_0, d_0) \leftarrow \setup(1^\lambda, D); \;\; \forall \, i:
(\tilde D_i, {\sf fetched}_i, \pi_i) \leftarrow \prove(\tilde{D}_{i-1}, I_i);
(b_i, d_i) \leftarrow \vrfy(d_{i-1}, I_i, {\sf fetched}_i, \pi_i),\]
then $b_1 = \cdots = b_m = 1$ and ${\sf fetched}_1, \ldots, {\sf fetched}_m$ are all
correct (as defined above).

Security requires that for all poly-time $\A$, any initial array $D$, and any
(adaptively chosen) sequence of instructions $I_1, \ldots, I_m$, if we run
\[(\tilde D_0, d_0) \leftarrow \setup(1^\lambda, D); \;\;\; \forall \, i:
(\tilde D_i, {\sf fetched}_i, \pi_i) \leftarrow \A(\tilde{D}_{i-1}, I_i);
(b_i, d_i) \leftarrow \vrfy(d_{i-1}, I_i, {\sf fetched}_i, \pi_i),\]
then the probability that $b_1 = \cdots = b_m = 1$ but ${\sf fetched}_m$ is not the correct answer
is negligible.

If the above holds even when $\A$ is given $d_0$, then we say the scheme is {\sf publicly verifiable}.
\end{definition}




\ignore{
Our verifiable-only VC-RAM construction will rely on standard,
publicly verifiable memory checking -- also referred to as the reliable
memory model in memory checking literature
\elaine{not sure what the term is, double check, and put in citation}, in
the sense that the client state (i.e., digest) need not
be kept secret from the server.
While previous works have given formal definitions for memory checking,
we observe that memory checking
is a special case of VC-RAM, where the RAM program is simple
memory read or memory write.
Therefore, we do not specially repeat the security definitions
for memory checking, but simply define it as a special case of VC-RAM.
\jnote{I'm not sure how useful it is to define memory checking as a special case of VC-RAM.
The definition is self-contained enough that we could just give it.} \jnote{We used to have
a short, self-contained definition; not sure what happened to it.}

\begin{definition}[Memory checking]
A memory checking scheme is a special case of VC-RAM, where the
RAM program either fetches memory location $a$ upon input
$(\rd, a)$, or
writes data $v$ to memory location~$a$ upon input
$(\wt, a, v)$.
For secretly verifiable memory checking, its security definition
follows from
Definition~\ref{defn:verifiable} -- note that memory checking
requires only verifiability but not privacy.
For publicly verifiable memory checking, the security definition
is similar to
Definition~\ref{defn:verifiable}, with the exception that the adversary
is also given the updated client state (denoted $\cst$) in lines marked [*] in
the security game.
%rivacy is not a requirement of memory checking,
\end{definition}
}

\subsection{Succinct Non-Interactive Arguments of Knowledge (SNARKs)}
\label{app:SNARKs}
\begin{definition}[SNARK]
%\jnote{To be a SNARG, the definition has to say something about the verification time relative
%to checking membership in~$R_L$.}
Algorithms $(\algKeygen, \P, \V, \Ex)$
give a {\sf succinct non-interactive argument of knowledge (SNARK)}
for an NP language $L$
with corresponding NP relation \jnote{Properly define.} $R_L$~if:
%\jnote{Use $\pi$ for proof instead of~$\sigma$?}
\begin{description}
\item
\textbf{Completeness:}
For all $x \in L$ with witness $w \in R_L(x)$:
\[
\Pr\left[\V(\sk, x, \pi) = 0 \ \bigg|
\begin{array}{c}
(\pk, \sk) \leftarrow \algKeygen(1^\lambda), \\
\pi\leftarrow \P(\pk, x, w)
\end{array}
\right] = \mathsf{negl}(\lambda)
\]
\item
\textbf{Adaptive soundness:}
For any probabilistic polynomial-time algorithm $\algA$,
\[
\Pr\left[
\begin{array}{c}
\V(\sk, x, \pi) = 1  \\
\wedge \ (x \notin L)
\end{array}
\ \bigg|
\begin{array}{c}
(\pk, \sk) \leftarrow \algKeygen(1^\lambda), \\
(x, \pi) \leftarrow \algA(1^\lambda, \pk)
\end{array}
\right] = \mathsf{negl}(\lambda)
\]

\item
\textbf{Succinctness:}
The length of a proof is given by
$|\pi| = \poly(\lambda) \poly \log (|x|+|w|)$.
\item
\textbf{Extractability.}
For any poly-size
prover $\P^*$, there exists an extractor $\Ex^*$
such that for any statement $x$, 
auxiliary information 
$\mu$, the following holds:
\[
\Pr\left[
\begin{array}{l}
(\pk, \sk) \leftarrow \algKeygen(1^\lambda)\\
\pi \leftarrow \P^*(\pk, x, \mu)\\
\V(\sk, x, \pi) = 1
\end{array}
\wedge
\begin{array}{l}
w \leftarrow \Ex^*(\pk, \sk, x, \pi) \\
w \notin R_L
\end{array}
\right] = {\sf negl}(\lambda)
\]
%For any statement $x$, %there exists an extractor algorithm $\Ex_x$,
%%such that
%for any $\pi \leftarrow \P(\pk, x, w)$,
%$w \leftarrow \Ex(\sk, x, \pi)$.
\ignore{
\item
\textbf{Zero-knowledge}
There exists a simulator $\SIM$,
such that for any polynomial-time adversary $\algA$,
it holds that
\[
\begin{array}{l}
\Pr[\pk \leftarrow \algKeygen(1^\lambda);
(x, w) \leftarrow \algA(\pk);  \pi \leftarrow \P(\pk, x, w):
(x, w) \in R \text{ and } \algA(\pi) = 1] \simeq\\
\Pr[(\pk, \state) \leftarrow \SIM(1^\lambda);
(x, w) \leftarrow \algA(\pk);  \pi \leftarrow \SIM(\pk, x, \state):
(x, w) \in R \text{ and } \algA(\pi) = 1]
\end{array}
\]
}
\end{description}
\end{definition}

%\paragraph{Public vs. secret verifiability.}
We say that a SNARK is {\it publicly
verifiable} if $\sk = \pk$.
In this case, proofs can be verified by anyone with $\pk$.
Otherwise, we call it a {\it secretly-verifiable} SNARK, in which
case only the party with $\sk$ can verify.

\begin{lemma}[Efficient SNARKs~\cite{spanprogram}]
Assume that the $q$-PDH assumption
and the $q$-PKE assumption
hold in an appropriately chosen bilinear group.
There exists a publicly verifiable SNARK
for Circuit-SAT where circuits have size at most~$s$,
such that
$\algKeygen$ takes $\tilde{O}(s) \cdot O(\lambda)$ time,
prover computation takes $\tilde{O}(s) \cdot O(\lambda)$ time,
verifier computation is
$O(|x| \lambda)$, the size of $\pk$ is
$O(s\lambda)$,
and proof size is~$O(\lambda)$.
Furthermore, assuming the $q$-PDH, $d$-PKE and
$q$-PKEQ assumptions,
there is a secretly verifiable SNARK
with the same asymptotic efficiency.
\end{lemma}


%\begin{proof}(of Theorem~\ref{thm:vc})
%Suppose the client starts with digest $d_i$, and the server starts
%with data $D_i$, digest $d_i$, and auxiliary authentication information $\aux_i$.
%\end{proof}





\subsection{Verifiable RAM Computation}
\begin{definition}[Verifiable RAM Computation]
A Verifiable RAM Computation (VC-RAM) scheme
consists of the following algorithms:
\label{defn:vcram}
\label{defn:correct}
\end{definition}

%In particular, the client's state is updated
%after running each client-side algorithm, including $\algSetup$,
%$\algProbgen$, and $\algVerify$.
\begin{description}
\ignore{
\item
$(\pk, \sk) \leftarrow \algKeygen(1^\lambda, 1^n, 1^\ell, 1^{\ell'})$:
The key-generation algorithm takes as input a security parameter $\lambda$,
size of memory $n$, bit length of each memory word $\ell$,
and bit length of input $\ell'$,
and outputs public key $\pk$ and secret key~$\sk$.
\jnote{What is the point of separating KeyGen and Setup?}
}
\item
%$(\overline{D}_0, \cst) \leftarrow \algSetup(\pk, \sk, D_0)$:
$(\sst, \cst) \leftarrow \algSetup(1^\lambda, D, \params)$:
The $\algSetup$ algorithm
\ignore{
\footnote{
One can alternatively split the the $\algSetup$ algorithm
into a $(\pk, \sk) \leftarrow \algInit(1^\lambda, 1^n, 1^\ell, 1^{\ell'})$
algorithm which outputs a public key $\pk$ and secret key $\sk$;
and a $(\sst, \cst) \leftarrow \algSetup(\pk, \sk, D)$
algorithm. In the security definition (Definition~\ref{defn:indist}),
the adversary would then be able to choose the initial memory $D$
dependent on the public key.
Our construction can also be proven secure under this slightly stronger model.
However, in practice, the client can always choose $\pk$ on the fly
whenever the initial $D$ is outsourced to the server.
We therefore go with simpler notations.
}
}
is a one-time setup algorithm run by the client.
$\algSetup$ takes in %the description of the RAM program $f$,
the security parameter $1^\lambda$,
initial memory array $D$, %(we assume that initial CPU states are always zeroed out),
%initial memory array $D[1..n]$
and parameters $\params: = (n, \ell, |\st|)$ of the RAM;
%$1^n, 1^\ell, 1^{\ell'}$, representing
%the length of the memory array,
%the bit length of each memory word,
%and the bit length of the input respectively.
and outputs
%the encoded memory content $\overline{D}_0$,
%a public key $\pk$,
server initial state $\sst$,
and client state $\cst$.
The client hands $\sst$ to the server,
and retains state $\cst$ locally.
%The client hands $\pk$ and $\overline{D}_0$ to the server.
%In a one-time setup phase, the client encodes the initial memory
%array $D_0[1..n]$, containing a description of the RAM program $f$ (i.e., code)
%as well as initial data, and outsources
%the resulting
%$\overline{D}_0$ to the server.
%The client retains state $\cst$ locally.
%In general, we can assume that the client state
%$\cst$ now contains $\pk$ and $\sk$, so we do not
%need to repeat $\pk$ and $\sk$ in the algorithms below.
%\elaine{whether this state can be made public decides whether
%this scheme is publicly or privately verifiable}
%The client and the server both run the $\algSetup$ algorithm,
%obtaining a concise digest $d_0$ of the memory contents including $f$ and $D$,
%as well as auxiliary information $\aux_0$.
%The client keeps the digest $d_0$. The server will later use $\aux_0$ to efficiently
%compute the updated digest when the memory contents get updated during the RAM computation.
\item
$(\overline{\inp}, \cst) \leftarrow \algPrepare(\inp, \cst)$:
Given input $\inp$, prepare the input and obtain the encoding
$\overline{\inp}$.
The client state $\cst$ is updated\footnote{
We use the notation $({\it output}, \state) \leftarrow {\sf Alg}({\it input}, \state)$
to denote that a party runs algorithm {\sf Alg}, which
results in updating its local state.
}.
\item
%$(\overline{\res}_i, \overline{D}_i)  \leftarrow \algCompute(\pk, \overline{x}_i, \overline{D}_{i-1})$:
$(\overline{\res}, \sst)  \leftarrow \algCompute(\overline{\inp}, \sst)$:
Given %public key $\pk$,
%encoded memory array $\overline{D}_{i-1}$,
current server state $\sst$
and encoded input $\overline{\inp}$, the server
computes an encoded answer $\overline{\res}$, which typically embeds the output
as well as a proof of correct computation.
The server also updates its state $\sst$ as a result of $\algCompute$.
\jnote{Be consistent whether $D$ or $\inp$ comes first.}
\item
$(\res, b, \cst) \leftarrow \algVerify(\overline{\res}, \cst)$:
Outputs the decoded answer $\res$, a bit $b \in \{0, 1\}$ indicating
whether the answer is accepted, and updates the client local state
$\cst$.
%\item
%$\{0, 1\} \leftarrow \algCheck(\pk, d, D)$:
%Check if $d$ is a correct digest of $D$, and output $0$ or $1$.
\end{description}
%\label{defn:vcram}

Correctness is defined in the obvious way. We require that for any
$\params$,
%$n, \ell, \ell' = \text{poly}(\lambda)$,
for any initial memory array $D \in \{0, 1\}^{\ell n}$,
for any query sequence $\inp_1, \inp_2, \ldots, \inp_m$ where
$m = \text{poly}(\lambda)$ \jnote{Should $\ell, \ell = O(\log \lambda)?$},
\[
\Pr\left[
\exists i:
\begin{array}{l}
(\res_i \neq f(D, \inp_1, \inp_2, \ldots, \inp_i))\\
\vee (b_i = 0)
\end{array}
\left|
\begin{array}{l}
%(\pk, \sk) \leftarrow \algKeygen(1^\lambda, 1^n, 1^\ell, 1^{\ell'})\\
(\sst_0, \cst)\leftarrow \algSetup(1^\lambda, D, \params)\\
%(\overline{D}_0, \cst)\leftarrow \algSetup(\pk, \sk, D_0)\\
\forall i \in \{1, 2, \ldots, m\}:\\
\quad \quad (\overline{\inp}_i, \cst) \leftarrow \algProbgen(\inp_i, \cst)\\
\quad \quad (\overline{\res}_i, \sst_i) \leftarrow
\algCompute(\overline{\inp}_i, \sst_{i-1})\\
\quad \quad (\res_i, b_i, \cst) \leftarrow \algVerify(\overline{\res}_i, \cst)
\end{array}
\right.
\right]
= \mathsf{negl}(\lambda)
\]

%\aish{I have defined Probgen and Verify to take secret key as input. If we want public verifiability or anyone to be able to delegate, we can consider definitions (defined in the natural way) with pk also. }
%\elaine{discuss public vs.\ secret key setting. the non-private setting,
%we can do public key with public-key SNARGs.}
%\paragraph{Public vs. secret key setting.}
%We say that a VC-RAM scheme is in the public key
%setting if the client state $\cst$ may be revealed to the server.
%We say that a VC-RAM scheme is in the secret key
%setting, if the client state $\cst$ must be kept secret from the server.

\ignore{
\paragraph{Correctness.}
[definition trivial]

\paragraph{Soundness.}

\paragraph{Privacy.}
input/output privacy: run setup and give public key to adversary.
adversary can choose client inputs for Setup
and Probgen.
but can also query a version of those algorithms where
the client chooses those inputs.
adversary can selectively ask to see
some outputs for Verify.
Define whatever information adversary learns in this way.

Privacy: there exists
a simulator (simsetup, simprobgen, simverify algorithms sharing state)
which has oracle access to the answer of each query.
%a simulator that can
%simulate the query answers knowing only
%[essentially what the adversary knows],
%and the [secret key], [how about secret state]?
%\elaine{need to make more precise here. the problem is that
%what the simulator knows is determined dynamically online}
such that the adversary does not know
whether it is talking to a simulator or
in real-world.
}

\paragraph{Server efficiency and client efficiency.}
We say that a VC-RAM scheme is {\it verifier efficient}, if
the client's online computation per query (including
the cost of the $\algProbgen$ and $\algVerify$ algorithms)
is asymptotically smaller than the run-time of the
RAM program.
In our constructions, the client's online computation
is independent
of the time $\tau$ of running the RAM program and the memory size $n$.
We say that a VC-RAM scheme is {\it server efficient},
if the server's online computation per query is sublinear in the
size of $D$ for queries that take sublinear time to execute
in the RAM model.
%We say that a VC-RAM scheme is efficient, if the client's online
%computation cost
%\end{definition}

\paragraph{}

We now define the security of VC-RAM.
We give two security definitions:
1) what is a {\it verifiable} VC-RAM scheme; and 2) what is
a {\it verifiable} and {\it private} VC-RAM scheme.
In both definitions, we
consider an honest client, and a {\it malicious} server.



\subsection{Security Definitions: Verifiability of VC-RAM}
\label{sec:defn-verifiability}
%\jnote{Why is this here? Again, does this handle the selective abort issue or not?}

\begin{definition}[Verifiable-only VC-RAM]
We say that a VC-RAM scheme is verifiable, if
for any polynomial time (stateful) adversary $\algA$ the following holds.
\label{defn:verifiable}
\end{definition}
\[
\Pr\left[
\begin{array}{l}
\exists i: (b_i = 1) \wedge   \\
(\res_i \neq f(D, \inp_1, \ldots, \inp_i) )
\end{array}
\left|
\begin{array}{l}
%(\pk, \sk) \leftarrow \algKeygen(1^\lambda, 1^n, 1^\ell, 1^{\ell'}) \\
%D_0 \leftarrow \algA(\pk, 1^n, 1^\ell, 1^{\ell'})\\
D \leftarrow \algA(\params)\\
(\sst, \cst) \leftarrow \algSetup(1^\lambda, D, \params) \\
\inp_1 \leftarrow \algA(\sst) \hfill [*]\\
\forall i \in  \{1, 2, \ldots, m\}: \\
%\quad (x_i, \mu) \leftarrow \algA(\mu)\\
\quad \quad (\overline{\inp}_i, \cst) \leftarrow \algProbgen(\inp_i, \cst)\\
\quad \quad \overline{\res}_i \leftarrow \algA(\overline{\inp}_i) \hfill [*]\\
\quad \quad (\res_i, b_i, \cst) \leftarrow \algVerify(\overline{\res}_i, \cst) \\
\quad \quad \inp_{i+1} \leftarrow \algA(\res_i, b_i) \hfill [*]
\end{array}
\right.
\right]
= \mathsf{negl}(\lambda)
\]

\paragraph{Public vs.\ secret verifiability, two-party vs.\ three-party settings.}
In the above, we have defined a secretly verifiable VC-RAM, i.e.,
the client state $\cst$ needs to be kept secret from the server.

We say that a VC-RAM scheme is {\it publicly verifiable} if the client
state $\cst$ necessary for the verification need not be kept secret from
the server.
More formally, in lines marked [*] in the above security definition,
we supply the latest client state $\cst$ to the adversary $\algA$ as well.

%\paragraph{Remark.}
Public verifiability is also referred to the {\it three-party} setting
in the authenticated data structure literature.
Imagine the scenario
%Basically, imagine
where a trusted data source (i.e., client)
distributes a signed copy of the latest client state $\cst$ (often
referred to as a digest in the authenticated data structure literature).
We can easily augment the definition to distinguish between two types of queries --
{\it read-only} queries and {\it update} queries.
Update queries (e.g., insertions and deletions to a database)
should only be made by the trusted data source,
and would write to memory as a result of the RAM execution.
By contrast, a read-only query (e.g., performing a binary or keyword search)
does not update the memory during the RAM execution,
and anyone with the latest client state (i.e., digest) $\cst$ should be able to issue read-only queries and
verify the result.
More formally, the $\algProbgen$ and $\algVerify$ algorithms for read-only queries
should not cause updates to the client state (i.e., digest) $\cst$.
How to enforce such access control is an orthogonal issue, and outside the scope of this paper.
%then anyone with $\cst$ would be able to verify queries.

\section{VC-RAM Construction}
\label{sec:verifiableonly}
In this section we sketch a construction for verifiable-only VC-RAM construction.
%in the secretly verifiable setting.
Our construction is based on any publicly-verifiable memory-checking scheme and~SNARK; see
Appendices~\ref{app:MC} and~\ref{app:SNARKs}.
Our construction is publicly verifiable if the underlying SNARK is publicly verifiable;
else it is secretly verifiable.

%We note that in spite of our verifiable and private construction described
Although it may appear as if our verifiable and private construction  
(Section~\ref{sec:VP}) subsumes the verifiable-only scheme,
the verifiable-only construction is actually  
worthy of independent interest, since 
1) we can additionally achieve public verifiability in this setting;
2) the verifiable-only scheme is cheaper than our verifiable and private
scheme -- it turns out that this is necessary 
for our \name application (Section~\ref{sec:vosbrief}). 

A memory-checking scheme easily yields an interactive VC-RAM scheme.
Say we want to execute queries starting from initial data array~$D$ (recall that $D$ includes both the
program~$f$ as well as any underlying data).
The client simply outsources storage of $D$ to a server using a memory-checking scheme.
To answer query~$x$, the client beings running the RAM program as in Section~\ref{sec:def-RAMs},
making read/write requests
to the
server as needed during the course of this execution.
Each time the server gives a response, the client first verifies the response (halting execution if
verification fails), and then updates its digest appropriately.
It is trivial to see that the resulting scheme satisfies verifiability.

The above approach can be made non-interactive if the memory-checking
scheme is publicly verifiable -- by having the client simply send
$x$ to the server, and then having the server
simulate on its own the actions of the client and the server
from the previous, interactive protocol. At the end, the server
sends the final result back to the client along with the entire sequence of proofs
that the client can verify all at once.

We describe this non-interactive scheme more formally since we will further modify it later below.
Let $(\setup, \prove, \vrfy)$ be a memory-checking scheme.
Given initial array~$D$ containing (in addition to data)
a program $f$ that runs for exactly~$\tau$
steps, the client
runs $(\tilde D_0, d_0) \leftarrow \setup(1^\lambda, D)$ and sends $\tilde D_0, d_0, \tau$
to the server; the client stores only~$d_0$.
The client also sends to the server a description of 
the $\nextins$  circuit.
When the client later wants to compute the answer to some query~$x$, it sends $x$ to the server.
The server sets $\st_0=x$ and ${\sf fetched}_0=\perp$. Then for $t=1, \ldots, \tau$ it
does:
\[
\begin{array}{l}
\left(\data_t, \waddr_{t}, \raddr_{t}, \st_{t}\right)
  := \nextins\left({\sf fetched}_{t-1}, \st_{t-1}\right)\\
I_t := (\data_t, \waddr_t, \raddr_t) \\
(\tilde D_t, {\sf fetched}_t, \pi_t) \leftarrow \prove(\tilde{D}_{t-1}, I_t)
\end{array}
\]
Finally, it sends the result $y=\data_\tau$ along with the sequence
$({\sf fetched}_1, \pi_1, \ldots, {\sf fetched}_\tau, \pi_\tau)$ to the client.
To verify, the client sets $\st_0=x$ and ${\sf fetched}_0=\perp$. Then for $t=1, \ldots, \tau$
it does:
\[
\begin{array}{l}
\left(\data_t, \waddr_{t}, \raddr_{t}, \st_{t}\right)
  := \nextins\left({\sf fetched}_{t-1}, \st_{t-1}\right)\\
I_t := (\data_t, \waddr_t, \raddr_t) \\
(b_t, d_t) \leftarrow \vrfy(d_{t-1}, I_t, {\sf fetched}_t, \pi_t)
\end{array}
\]
If $b_1 = \cdots = b_\tau = 1$ and $y=\data_\tau$ then the client accepts~$y$ as the correct result.
The client updates its local digest to~$d_\tau$; by doing so, the client
and server are now ready to evaluate the RAM program again on some other input~$x'$ (with the memory
array updated appropriately).


%\elaine{todo: use a constant overhead memory checking scheme --
%it does not circumvent the memory checking lower-bound?
%need to double check}

The above scheme allows the client to outsource \emph{storage}, but not \emph{computation}.
We can address this, however, using a SNARK. Define the following $\mathcal{NP}$
relation~$R$:
\[\left( \rule{0pt}{10pt} (x, y, d_0, d_\tau), \; ({\sf fetched}_1, \pi_1, \ldots, {\sf fetched}_\tau, \pi_\tau) \right) \in R\]
iff the client verification described above succeeds, and furthermore the final
state of the client's digest is~$d_\tau$.
We now modify the previous scheme as follows:
Rather than having the server send $({\sf fetched}_1, \pi_1, \ldots, {\sf fetched}_\tau, \pi_\tau)$,
and then having
the client verify all these proofs, we instead have the server \emph{prove} (using a SNARK)
that a valid proof sequence exists. Finally, the server sends 
the client $(y, d_\tau)$, in addition to a succinct proof 
that a valid witness 
$({\sf fetched}_1, \pi_1, \ldots, {\sf fetched}_\tau, \pi_\tau)$ exists
for the NP statement $(x, y, d_0, d_\tau)$.
Using the SNARK from~\cite{spanprogram} (see Appendix~\ref{app:SNARKs}) and
any memory-checking scheme, we thus obtain:




\begin{thm}[Efficient Verifiable RAM Computation]
\label{thm:vc}
The above gives a verifiable (but not private) 
VC-RAM where the server runs in time $O(\tau \log n) \cdot \poly(\lambda)$,
the verifier runs in time $O(|x| \cdot \lambda)$, and the proof size is~$O(\lambda)$.

The VC-RAM is publicly verifiable if the memory-checking scheme and SNARK are.
\end{thm}
\begin{proof}
The correctness of the construction can be derived from the correctness
of the underlying SNARK and memory checking scheme in a straightforward
manner.

Verifiability of the client-efficient construction follows from verifiability
of the client-inefficient construction plus extractability of the SNARK; namely,
if there exists an adversary who can break verifiability in the client-efficient version,
then by extracting from that adversary a witness ${\sf fetched}_1, \pi_1, \ldots, {\sf fetched}_\tau, \pi_\tau$
we can break verifiability of the client-inefficient version.
We defer the straightforward details to the full version.
\end{proof}


\ignore{
Due to the soundness of the underlying SNARG,
except with negligible probability,
there exists an admissible trace $\tr$ which satisfies the above properties
V1, V2, V3, and V4.
Now, due to the soundness of the memory checking scheme,
all data values read from memory in the trace $\tr$ must be correct.
Therefore, we have an execution trace such that
1) it terminated correctly in $\tau$ time steps (V1);
2) its initial and terminating states are consistent
with the inputs and outputs (V4);
3) every state transition function is performed correctly (V2);
and 4) every memory read returned the correct value (V3).

\elaine{memory checking scheme works across multiple queries.}
}

\ignore{
\section{Enforcing Honest Server Behavior}
In this section, we show how to apply Verifiable Computation for RAM programs
to achieve security against a malicious
server, and enforce that the server adheres to the prescribed computation.
We describe our scheme based on
the binary ORAM~\cite{asiacrypt11}.

\subsection{Scheme}
\elaine{below is a sketch, needs to be spelled out, and presented more formally.}

During a one-time initialization stage, the client
runs the $\mathsf{VCRAM}.\algKeygen(1^\lambda)$ algorithm to obtain
a public key $\mathsf{VCRAM}.\pk$,
and runs the FHE key setup algorithm \elaine{use notation}
to obtain $(\mathsf{FHE}.\pk, \mathsf{FHE}.\sk)$.
The client sends
$\pk := (\mathsf{VCRAM}.\pk, \mathsf{FHE}.\pk)$ to the server.
The client uses FHE to prepare initial ORAM image.
The client uploads the FHE-encrypted ORAM image as well as the RAM program $f$
for accessing the ORAM to the server.
The client runs the $\mathsf{VCRAM}.\algSetup$ algorithm on the FHE-encrypted ORAM image
and $f$,
and keeps an initial digest $d_0$ locally.

Each data request consists of $O(\log N)$ rounds of communication, where each round accesses
a binary tree~\cite{asiacrypt11}.
In each round, the following happens:
\begin{itemize}
\item
\rr stage:
\begin{itemize}
\item
Client sends request to server, the request contains a path identifier (referred to
as the designated leaf node in \cite{asiacrpt11}),
an FHE-encrypted block identifier.
\item
Server performs $\text{poly} \log(N)$ amount of RAM computation,
and sends an FHE-encrypted block to the client.
A total of $\text{poly} \log(N)$ blocks along the path are touched and updated on the server.
The server sends the updated digest to the client.
\item
Using the $\algVerify$ algorithm of the VC-RAM scheme, Client verifies the
correctness of the block returned as well as the updated digest.
\end{itemize}
\item
\add stage:
\begin{itemize}
\item
Client sends FHE-encrypted block identifier and contents to the server.
Client also sends the server $O(\log(N))$ random numbers indicating which buckets
at each level to evict from.
\item
Server adds the block to the root bucket using operations under FHE.
Server performs eviction using operations under FHE.
As a result of the computation, $\text{poly} \log(N)$ blocks on the server
are updated.
Server sends client the updated digest.
\item
Using the $\algVerify$ algorithm of the VC-RAM scheme, Client verifies the
correctness of the updated digest.
\end{itemize}
\end{itemize}



\subsection{Proof of Security in the Malicious Model}

\elaine{need to unify notation globally}
\begin{proof}(malicious model.)

For any real-world adversary $\algS$, we show how to build a simulator $\algSbar$ in the ideal world,
such that the joint distribution of the two parties in the ideal world are indistinguishable
from the real world.


\elaine{to fix: simulator has VC-RAM.sk}
\paragraph{Init.}
\elaine{to fix: adversary can choose D0}
The simulator $\algSbar$ generates the public key $\pk := (\mathsf{VCRAM}.\pk, \mathsf{FHE}.\pk)$,
\elaine{define keygen algorithm}
and gives them to the real-world adversary $\algS$.
$\algSbar$ now prepares an FHE-encrypted ORAM image -- initially, all memory locations
contain zero.
$\algSbar$ now runs
the $\mathsf{VCRAM}.\algSetup$
algorithm on the RAM program $f$ and
on the FHE-encrypted ORAM image,
and records the current digest $d_0$.
$\algSbar$ sends $\algS$ the encrypted ORAM image and $f$.

%On receiving the public key
%the RAM program $f$, and the encrypted ORAM image from the client,
%$\algSbar$ sends an $\mathsf{init}$ message to the trusted third-party $\T$.

\paragraph{Query.}
At any point of time, $\algSbar$
always keeps track of the latest digest $d_i$.

Whenever $\algSbar$ receives a message $\bot$ from the trusted third-party $\T$,
It performs $O(\log N)$ rounds of communication with $\algS$.
For each round of communication:
$\algSbar$ sends a ``random'' request to $\algS$.
\elaine{elaborate on random}
%and performs $\text{poly} \log (N)$ amount of RAM computation.
If $\algS$ aborts during any round, $\algSbar$ sends $0$ to $\T$.
Otherwise,
$\algSbar$ checks if the output of $\algS$ in every round passes the $\mathsf{VCRAM}.\algVerify$
algorithm (using the recorded digest).
If $\mathsf{VCRAM}.\algVerify$ fails, $\algSbar$ sends $0$ to $\T$.
Otherwise,
If $\mathsf{VCRAM}.\algVerify$
passes for all rounds,
$\algSbar$ updates its recorded digest, and sends $1$ to $\T$.

%\paragraph{Output.}


\paragraph{Indistinguishability of real-world and ideal-world execution.}
First, due to the semantic security of the FHE encryption scheme
and the security of the underlying ORAM,
the real-world server $\algS$ cannot distinguish whether it is talking
to a real-world client with the same input $x$, or talking to a simulator $\algSbar$ -- otherwise,
it is easy to build a reduction showing that one can leverage $\algS$ to break
the semantic security of the FHE scheme or the security of the underlying ORAM scheme.

Now, we use a sequence $(b_1, b_2, \ldots, )$ of indicator bits
to denote whether
$\algS$'s outputs pass the $\mathsf{VCRAM}.\algVerify$
algorithm with each data request. $b_i = 1$ means that $\algS$'s outputs
pass the
$\mathsf{VCRAM}.\algVerify$
algorithm for all communication rounds for the $i$-th data request; otherwise $b_i = 0$.
This sequence of indicator bits
must be computationally indistinguishable whether $\algS$ is talking to a
real-world client with input $x$, or an ideal-world simulator $\algSbar$ -- since otherwise,
one could easily leverage $\algS$
to break the semantic security of the FHE scheme or the security of the underlying ORAM.

Now, assume that the VC-RAM scheme is secure.
In the real world, if the server's messages all pass the
$\mathsf{VCRAM}.\algVerify$ algorithm, then the results returned must match the results
obtained from correctly following the prescribed RAM program.
Therefore, it is not hard to see that
$\textsc{Ideal}_{\algCbar(x),\algSbar(y)}$ must be computationally indistinguishable
from $\textsc{Real}_{\algC(x),\algS(y)}$ -- otherwise,


%Note:
%Indistinguishability of
%$\textsc{Ideal}_{\algC(x),\algS(y)}$
%and
%$\textsc{Real}_{\algC(x),\algS(y)}$
%relies on
%1) security of underlying ORAM;
%2) semantic security of FHE scheme;
%and 3) security of the VC-RAM scheme.
%\elaine{continue proof}
\end{proof}
}



%\section{Application: Verifiable Oblivious Storage} 
%\elaine{it is a generalization of blah}

%\subsection{\name with $O(\log n)$ Bandwidth Overhead and $O(1)$ Client Memory, Secure in the Malicious Model}
\section{\namebig}
\label{sec:vos}
VC-RAM is a very powerful primitive, and has immediate implications in
several areas in cryptography.
One application is to build efficient \name schemes. 
Oblivious RAM (ORAM)~\cite{oram00,oram01,oram02,oram03,oram07,oram09,oram13} 
was initially intended for obliviously simulating RAM programs 
and achieving software piracy.
In this setting, the memory is considered 
{\it passive}, i.e., only
capable of fetching and storing data, but not capable of performing
computation.

As cloud computing gains popularity, ORAM was 
applied to the outsourced data setting, such that
a client can store data with an untrusted server,
such that the server learns nothing about the data or
access patterns. In this setting, it is natural to consider
{\it active} servers capable of performing computation. 
In fact, a couple recent constructions~\cite{oram14,LO12} 
have leveraged server-side computation to achieve single-round ORAM.
\ignore{
several constructions~\cite{oram14,LO12}.
Particularly, 
Williams \etal~\cite{oram14} and Lu \etal~\cite{LO12} 
show that by leveraging storage-side computation,
we can build single-round ORAM schemes.
}


\subsection{Complete Security Definition for \nameshort}
To differentiate, we use the terminology \name (\nameshort) 
to refer to the data outsourcing setting when the server
actively performs computation.
In fact, we observe that the \nameshort setting demands a new security
definition from the traditional ORAM setting, and the security
definition should take into account the fact that a dishonest
server can arbitrarily deviate from the prescribed computation (other
than simply corrupting data blocks as the model considered
by Goldreich and Ostrovsky in their initial ORAM work~\cite{oram00}),
We note that our VC-RAM security definition immediately implies 
a full-fledged security definition for \nameshort, since
\nameshort is just special case of VC-RAM with 
a degenerated RAM program that simply reads or
writes data.


\subsection{More Efficient \nameshort Construction}
Under the traditional ORAM (i.e., passive memory) setting, 
with $O(1)$ %(in terms of data blocks) 
client private memory,
the best known result is a recent scheme 
by Kushilevitz, Lu, and Ostrovsky~\cite{oram03},
achieving $O((\log n)^2/ \log\log n)$ bandwidth overhead. 
Meanwhile, lower bound results~\cite{oram00,oramlower}
have suggested that 
there is an $O(\log n \log \log n)$
lower bound on the bandwidth overhead of any Oblivious RAM scheme.

%\paragraph{Summary of result.} 
However, we observe that this super-logarithmic lower bound
does {\it not} apply to the \nameshort setting.
%In fact, we propose a novel \nameshort construction  
%achieving only $O(\log n/\log \log n)$ 
%bandwidth overhead, and secure against a malicious server.
By leveraging server-side computation,
we can build \nameshort
schemes that are more bandwidth efficient than
traditional ORAM schemes, and meanwhile, ensure
verifiability against an arbitrarily malicious server.

%\begin{thm}
%Assume 1) existence of collision resistance hash functions, 
%\elaine{fill in ring LWE blah, assumption of SNARK}.
%Then, there exists a \nameshort scheme for a reasonably
%large block size $\beta > $ \elaine{fill in},
%with $O(\log n / \log \log n)$
%bandwidth overhead, and $\poly (\log n, \lambda)$ server computation 
%(per data access), where $n$ is the total number of blocks
%and $\lambda$ is the security parameter.
%\end{thm}

\begin{thm}
Let $g(n)$ denote some function on $n$.
Assume collision resistance hash functions, 
the ring LWE assumption with 
suitable parametrization~\cite{fhe10,BGV12},
and the $q$-PDH and $q$-PKE assumptions~\cite{spanprogram}.
Then, there exists a \nameshort scheme for a reasonably
large block size\footnote{
Note that the \nameshort data block size $\beta$ is differet from the
memory word size $\ell$ of the RAM.
We assume $\beta = \tilde{\Omega}(\lambda)$, and 
$\ell = O(\log \lambda)$.
}
 $\beta = \tilde{\Omega}(\lambda)$,
%with $O(\log n / \log \log n)$
with $O(\log n / \log g(n))$ 
bandwidth overhead, and 
%$\poly (\log n, \lambda)$ server computation 
$O(g(n) \log^2 n / \log g(n))\poly(\lambda)$ server computation 
(per data access), where $n$ is the total number of blocks
and $\lambda$ is the security parameter.
\label{thm:vos}
\end{thm}

The following table shows some interesting special cases
of Theorem~\ref{thm:vos}.

\begin{center}
{
\renewcommand{\arraystretch}{1.3}
\begin{tabular}{c|c|c}
$g(n)$ & server computation & bandwidth overhead \\
\hline 
$n^{\alpha}$ for constant $\alpha < 1$ & $O(n^\alpha \log n)\poly(\lambda)$ & $O(1)$\\
\hline
$(\log n)^{(\log \log n)^\alpha}$ & 
\multirow{2}{*}{$O((\log n)^{ (\log \log n)^\alpha + 2}/(\log \log n)^{\alpha+1} )\poly(\lambda)$} & \multirow{2}{*}{$O(\log n/(\log \log n)^{\alpha+1})$}\\
for constant $\alpha \geq 0$  & & \\
\hline
$\log n$ &  $O(\log^3 n/\log \log n)\poly(\lambda)$ & $O(\log n / \log \log n)$ \\
\hline
constant $\alpha > 1$ &  $O(\log^2 n)\poly(\lambda)$ & $O(\log n)$
\end{tabular}
}
\end{center}




\ignore{Oblivious RAM was previously studied in the 
RAM computation model,
where the CPU is trusted, and RAM is untrusted.
The goal is to obfuscate the memory addresses visited
during the RAM computation, such that the memory addresses
accessed do not leak anything about memory contents (including
code and data). 
Formal definitions of ORAM in this model were originally given
by Goldreich and Ostrovsky in their pioneering 
paper~\cite{oram00} on ORAM.

Later, ORAM applications extended to the cloud
setting, where a trusted client would like to outsource
data storage to an untrusted cloud server. 
Earlier ORAM schemes assumed a passive server which supports
only read and write operations, and does not perform any
computation -- in this model, 
one can think of the server as being a remote disk.
This setting is similar to the RAM computation setting, where
the storage does not perform computation.
}

\ignore{
We make the following contributions: 
\begin{itemize}
\item
We observe that by having the storage actively perform 
computation, we can circumvent the known $O(\log n \log\log n)$ 
lower bounds on passive ORAM.
Particularly, we propose an \name scheme
that achieve $O(\log n)$
bandwidth overhead for reasonably large block sizes.
Particularly, our construction relies on the verifiable-only
VC-RAM scheme to ensure security against a 
malicious storage provider.
\item
While the active ORAM model have been implicitly considered
in several recent papers, 
we observe a full-fledged 
security definition in this setting has not been formally formulated,
especially when the storage can potentially be malicious.
Our security definitions for VC-RAM immediately gives
rise to a full-fledged security definition for \name,
since \name may be considered as a special case of VC-RAM,
where the RAM program simply reads or writes data.
\end{itemize}
}
\ignore{Essentially, these lower bounds do not apply in the model where 
the server can perform computation.
To distinguish from previous ORAM schemes which assume a passive server,
we call our construction \name. 
}

%An \name scheme is basically an 
%ORAM scheme
%except that the server now can perform active computation.
%\name can be thought of as a special case of verifiable 
%RAM computation -- since each data access can be thought
%of as a RAM program that simply reads or writes a block.

%Our verifiable RAM computation scheme immediately implies
%a single-round \name scheme with $O(1)$ bandwidth overhead,
%i.e., \elaine{define this metric.}

\ignore{Notice that previous work has (implicitly) leveraged
server computation to achieve single-round 
\name with polylogarithmic bandwidth overhead\elaine{cite two papers}.
However, they did not give a complete security formulation 
in this model.
}

\ignore{
\begin{table}
\centering
\begin{tabular}{l|c|c|c}
\hline
                  & storage-side computation & bandwidth overhead & model \\
\hline
Best known ORAM~\cite{oram13} ($\dagger$) & N/A   & $O(\log^2 n/\log\log n)$   & privately accessible\\
\hline
PIR & $\Omega(n)$ & $O(1)$ (*) & publicly accessible   \\
\hline
Our \name  &  $O(\text{poly}\log n)$  & $O(\log n)$ (*) & privately accessible\\ 
\hline
\end{tabular}
\caption{{\bf Comparison between ORAM and Private Information Retrieval (PIR).}\newline
(*): Costs are stated for reasonably large block sizes.
PIR with $O(1)$ bandwidth overhead can be achieved using FHE 
for reasonably large block sizes using ciphertext packing
for FHE schemes. \elaine{cite something}\newline
($\dagger$) We only consider ORAM schemes with $O(1)$ client private memory.
}
\label{tab:orampir}
\end{table}
}

\paragraph{Intuition.} Our main idea is as follows:
\begin{itemize}
\item
{\it Rely on Fully Homomorphic Encryption (FHE) to outsource
client computation to the server} whenever possible, and henceforth
reduce communication overhead between the client and server.
One technicality here is that to preserve bandwidth overhead, 
we need to use FHE ciphertext packing techniques such
that we can encrypt each data block of size 
$\beta$ using a $O(\beta)$-bit ciphertext, and still
maintain the ability to perform operations on each individual bit.
This can be achieved using techniques described in 
recent works~\cite{BGV12,BGH13}.
\item
{\it Use our verifiable-only VC-RAM construction 
(Section~\ref{sec:verifiableonly}) to enforce honest server
behavior}\footnote{
Although our verifiable and private VC-RAM construction implies
a \nameshort scheme,  it does not achieve the desired bandwidth
overhead. We need the verifiable-only VC-RAM scheme here because
it is cheaper than our verifiable and private VC-RAM scheme.
}.
\item
{\it Balance reads and writes to achieve better bandwidth overhead.}
If we apply the above FHE and VC-RAM idea 
to the ORAM construction by Goodrich and Mitzenmacher~\cite{oram09}, 
we can easily obtain a \nameshort scheme with 
$O(\log n)$ overhead. 
%Achieving $O(\log n/\log \log n)$ overhead
%is more tricky, and requires the balancing trick 
%proposed by Kushilevitz, Lu, and Ostrovsky~\cite{oram03}.

To achieve lower bandwidth overhead, we can use the 
the balancing trick 
proposed by Kushilevitz, Lu, and Ostrovsky~\cite{oram03}.
%to balance the cost of the read and write phases.
%The idea to make adjacent levels grow faster than a constant.
%For simplicity, we illustrate with an example where $g(n) = 2 \log n$. 
The idea is that for a scheme where reads and   
writes are not of equal cost, 
we can balance their cost to 
achieve better asymptotic bandwidth overhead.
Interestingly, while writes are typically more expensive than reads
in the non-FHE setting~\cite{oram09},  
it turns out that reads are 
more expensive (in terms of bandwidth overhead) 
under FHE, since the writes correspond
to shuffling operations which the server can homomorphically
evaluate on its own.
Therefore, by balancing reads and writes under FHE, we
are actually unbalancing them in the traditional non-FHE setting.

To balance reads and writes, we will adjust the rate (denoted $g(n)$)
at which adjacent levels in the storage hierarchy 
grows. When the next level grows faster, we get schemes with better
bandwidth overhead -- however, at the cost of more server computation.
\end{itemize}



\paragraph{Preliminary: the GM-ORAM scheme.}
As a starting point, consider the Oblivious RAM scheme
by Goodrich and Mitzenmacher~\cite{oram09}.
On a high level, their scheme~\cite{oram09} works as follows.
The server-side storage is divided into $O(\log n)$ levels 
denoted $B_k, B_{k+1}, \ldots B_L$, where $k$ is an appropriately
chosen initial starting point for the hierarchy.
Each level $B_i$ has capacity $2^i$.

For technical reasons related to proving 
inverse superpolynomial (i.e., negligible) failure
probabilities, levels $k+1, \ldots, K$ are treated
differently from levels $K+1, \ldots, L$, 
where $K = O(\log \log n)$.

\begin{itemize}
\item
$B_k$ is a table that the client scans through on every data access. 
\item
For a lower level $i \in \{k+1, \ldots, K\}$, 
$B_i$ contains
$2^{i+1}$ hash buckets, each of size $O(\log n)$.
\item
For an upper level $i \in \{K+1, \ldots, L\}$,
$B_i$ is a cuckoo hash table with $(1+\epsilon)\cdot 2^{i+2}$ cells,
and a stash of size $s = O(\log n)$.
\end{itemize}

The data access operations are sketched below. Each data access
request has a read and a write phase.

%\begin{figure}[h]
\noindent\begin{boxedminipage}{\textwidth}
\underline{{\bf Read phase~\cite{oram09}.}}
\begin{itemize}
\item
Scan through $B_k$, if block $\blockid$ is in there, 
$\found := $ true; else $\found := $ false.
\item
For each level $i \in \{k+1, \ldots, K\}$:
if $\found = $ true, read a random hash bucket;
else look for block $\blockid$ in hash bucket $B_i[h_i(\blockid)]$
in level $B_i$. If found, mark $\found := $ true.
\item
For each level $i \in \{K+1, \ldots, L\}$:

If $\found = $ true:  Read $B_i[h_{i, 0}({\sf nextdummy})]$ 
$B_i[h_{i, 1}({\sf nextdummy})]$, and 
let ${\sf nextdummy} \leftarrow {\sf nextdummy} + 1$.

Else:   Read $B_i[h_{i, 0}(\blockid)]$ 
$B_i[h_{i, 1}(\blockid)]$, and 
if found, mark $\found := $ true.

No matter which case: read through the stash at level $i$. 
If found, mark $\found := $ true.

\item
For all of the above: after reading any block, the block is reencrypted 
and written back. If the block is $\blockid$, mark
the block as obsolete before reencryption.
\end{itemize}
\end{boxedminipage}

\paragraph{}
\noindent\begin{boxedminipage}{\textwidth}
\underline{{\bf Write phase~\cite{oram09}.}}
\begin{itemize}
\item
If $B_k$ is not full: write the block $\blockid$ back to level $B_k$,
replace with updated block if necessary.
\item
If $B_k$ is full, find consecutively full levels
$B_k, B_{k+1}, \ldots, B_{m}$, such that $B_{m+1}$ is the first
empty level.
Reshuffle levels $B_k, \ldots, B_m$ into level 
$B_{m+1}$ -- this involves obliviously rebuilding 
a regular hash table or 
a cuckoo hash table at $B_{m+1}$ (see \cite{oram09} for 
the detailed algorithm)
New hashes are chosen every time for 
level being rebuilt (by choosing a new secret key freshly at random).
\end{itemize}
\end{boxedminipage}
%\label{fig:gm}
%\caption{Sketch of the GM-ORAM construction~\cite{oram09}.}
%\end{figure}

\paragraph{Step 1: Applying FHE.}
Suppose that all blocks are encrypted under FHE.
%In each data access query, the client sends the server
%$\overline{\blockid} \leftarrow \fhe.\enc(\blockid)$.
We now show how to execute the read and write phases more efficiently
by leveraging server-side FHE evaluations.

\noindent\begin{boxedminipage}{\textwidth}
\underline{{\bf Read phase under FHE.}}
\begin{itemize}
\item
{\it Client}: $\found := $ false.
$\overline{\blockid} \leftarrow \fhe.\enc(\blockid)$.
Send $\overline{\blockid}$ to server. 

\item
{\bf Level} $k$:

{\it Server}: 
$\overline{{\sf block}} \leftarrow \fhe.\eval(\find(\overline{\blockid}, \overline{B}_k))$.
where $\find$ is the function that looks for a block $\blockid$
in a table, and returns the block found or $\bot$ upon failure.
Send $\overline{{\sf block}}$ to client.

{\it Client}: 
Decrypt $\overline{{\sf block}}$ and update $\found$ appropriately. 


\item
{\bf For each level $i \in \{k+1, \ldots, K\}$}:

{\it Client}: if $\found = $ true, choose $a$ at random;
else choose $a \leftarrow h_i(\blockid)$.
Send $a$ to server. 

{\it Server:} 
$\overline{{\sf block}} \leftarrow \fhe.\eval(\find(\overline{\blockid}, \overline{B}_i[a]))$.
Send $\overline{{\sf block}}$ to client.

{\it Client}: 
Decrypt $\overline{{\sf block}}$ and update $\found$ appropriately. 

\item
{\bf For each level $i \in \{K+1, \ldots, L\}$}:

{\it Client}: if $\found = $ true, choose $a_0, a_1$ at random;
else choose $a_0 \leftarrow h_{i,0}(\blockid)$,
$a_1 \leftarrow h_{i,1}(\blockid)$.
Send $a_0, a_1$ to server. 

{\it Server:} 
$\overline{{\sf block}} 
\leftarrow \fhe.\eval(\find(\overline{\blockid}, \overline{B}_i[a_0], 
\overline{B}_i[a_1], 
\overline{B}_i[\text{stash}_i]))$.
Send $\overline{{\sf block}}$ to client.

{\it Client}: 
Decrypt $\overline{{\sf block}}$ and update $\found$ appropriately. 
\end{itemize}
\end{boxedminipage}
\paragraph{}

\noindent\begin{boxedminipage}{\textwidth}
\underline{{\bf Write phase under FHE.}}
\begin{itemize}
\item
{\it Client}: 
$\overline{{\sf block}} \leftarrow \fhe.\enc(\blockid, {\sf data})$.
Choose one more more fresh random hash keys.
$\overline{{\sf keys}} \leftarrow \fhe.\enc({\sf keys})$.
Send $\overline{{\sf block}}, \overline{{\sf keys}}$ to server. 
\item
{\it Server:}
If $B_k$ is not full (the server can know this by keeping
a counter of total data requests):
$\overline{B}_k \leftarrow 
\fhe.\eval({\sf insert}(\overline{B}_k, \overline{{\sf block}}))$
where ${\sf insert}$ is the function that
inserts a block into a table. 

If $B_k$ is full, find consecutively full levels
$B_k, B_{k+1}, \ldots, B_{m}$ (the server knows this by keeping
a counter of total data requests), such that $B_{m+1}$ is the first
empty level.
Homomorphically evaluate 
$\overline{B}_k, \ldots, \overline{B}_{m+1} \leftarrow 
\fhe.\eval({\sf reshuffle}(\overline{{\sf keys}}, \overline{B}_k, \ldots, \overline{B}_m))$ 
where ${\sf reshuffle}$
is the function that 
empties levels $B_k$ through $B_m$, and reshuffles them into level
$B_{m+1}$ -- suppressing obsolete blocks in the meanwhile.
\end{itemize}
\end{boxedminipage}

\paragraph{}
Using the above FHE idea, the read phase requires sending a single
FHE-encrypted block at every level, introducing $O(\log n)$ bandwidth
overhead, and $O(\log n)$ rounds of interaction.
The write phase requires no communication -- other than the client
sending to the server the FHE-encrypted new block and the necessary
FHE-encrypted hash keys. Basically, the server
is capable of performing reshuffling operations 
under FHE on its own.
The server has $O(\log^2 n) \poly(\lambda)$  
amortized computation overhead per data access 
in this scheme.



\paragraph{Step 2: Applying verifiable-only VC-RAM to enforce
honest server behavior.}
The above gives us 
an $O(\log n)$ overhead \nameshort scheme assuming a semi-honest server. 
However, the server may be malicious, and may not perform
the prescribed shuffling and homormorphic evaluations honestly. 

To address this problem, we can 
apply our verifiable-only VC-RAM
scheme. Observe that all the homomorphic evaluations
the server performs can be modelled as RAM computation. 
During the setup phase, the client will upload the 
FHE-encrypted and initially shuffled data blocks to the server,
and retain a digest of this initial storage snapshot.
Later in the online phase, the server can always 
use the verifiable-only VC-RAM 
to prove to the client that it has performed the prescribed
computation correctly, i.e.,  
1) the claimed result returned to the client is correct; and 
2) the claimed
new digest of the updated storage snapshot is correct.



\paragraph{Step 3: Balance reads and writes.}
In the above scheme, the read phase requires $O(\log n)$
overhead, while 
the write phase requires constant overhead. 
To balance reads and writes, 
we can make the next level grow faster than a constant
rate than the previous level -- however, while this reduces
bandwidth overhead, the tradeoff is server computation.

As an example, consider $g(n) = 2\log n$, i.e., we make level $B_{i+1}$ 
larger than level $B_i$ by $2 \log n$ times.
In this way, we have $O(\log n/\log \log n)$ levels.
Every time $B_k, \ldots, B_i$ are consecutively full, 
they will be shuffled into a subsequent non-full level 
$B_{i+1}$. Starting from an empty $B_{i+1}$, 
levels $B_k, \ldots, B_i$
will be shuffled into $B_{i+1}$  a total of 
$\log n$ times -- at which point $B_{i+1}$ is deemed full as well.
Every time $B_k, \ldots, B_i$ are shuffled into $B_{i+1}$, all existing
blocks inside level $B_k, \ldots, B_i$ and 
$B_{i+1}$ are shuffled together,
and written back to $B_{i+1}$.
%(note that this is different from 
%the scheme by Kushilevitz, Lu, and Ostrosky~\cite{oram03}).

It is easy to adjust the parameter $K$ (separating the lower
and upper levels)
correspondingly
such that the same failure probability analysis holds
as in the GM-ORAM scheme~\cite{oram09}.


It is not hard to see that with this new rebalanced scheme, 
the read phase now requires only $O(\log n /\log \log n)$
bandwidth overhead, while write phase requires constant 
bandwidth overhead.
The server has $O(\log^3 n)\poly(\lambda)$  
amortized computation overhead per data access 
in this rebalanced scheme.
It is also not hard to apply a similar deamortization
trick described in \cite{oram02} and \cite{oram03}, to spread
out the reshuffling work over time, such
that the server computation 
is $O(\log^3 n)$ per data access in the worst-case.
Note that our rebalancing is in the opposite direction
of the Kushilevitz, Lu, and Ostrovsky scheme~\cite{oram03}.
In the FHE setting, reads are more expensive; while
in the standard ORAM setting they consider, writes are more expensive.



More generally, we can set $g(n)$ to be other functions
as shown in the table in Theorem~\ref{thm:vos}.
For example, when $g(n) = \sqrt{n}$, 
the bandwidth overhead is $O(1)$ -- however,
server computation is $O(\sqrt{n} \log n)\poly(\lambda)$ per data access.
The scheme for $g(n) = \sqrt{n}$ is actually a 
variant of the Square-Root construction~\cite{oram00} -- 
with an adjusted hashing strategy to achieve 
inverse superpolynomial failure probability.


\ignore{
It is also interesting to compare \name 
with Private Information Retrieval (PIR) -- 
both can be used to securely fetch data 
from an untrusted cloud provider without leaking access patterns.
See Table~\ref{tab:orampir}.

\elaine{elaborate here.}

\paragraph{Active ORAM Construction.}

Consider the verifiable-only VC-RAM construction of~\ref{sec:nopriv-constr}, which allows the client to outsource the computation of some RAM program $\Pi$. The main idea of Active ORAM is to define the VC-RAM's $\Pi$ as a particular sequence (given in the sequel) of read/writes performed by the ORAM client in a typical ORAM scheme. We will also use fully homomorphic encryption to allow the server to perform each such sequence in one large block. For concreteness, we instantiate our Active ORAM using the Tree-based ORAM scheme of~\cite{SCSL11}.

Recall that in~\cite{SCSL11}, the ORAM is recursively defined as a sequence of $O(\log_c n)$ balanced binary trees, each of height $O(\log n)$, where each node of the tree is a trivial (``bucket'') ORAM of size $O(\log n)$. To perform one read/write of the original RAM program, the client reads/writes to every bucket ORAM along a single root-to-leaf path, for each tree of the recursive construction. We assume each block in the ORAM is encrypted ``on top'' by a layer of FHE.

Consider the sequence of reads/writes along \emph{one} of the $O(\log n)$ paths involved in one ORAM query. First, we construct a circuit for homomorphic evaluation with $O(\log^3 n)$ input wires, corresponding to each bit of the $O(\log^2 n)$ blocks in the given path, that computes ``one path's worth'' of a single ORAM query. We describe the topology of the desired circuit in terms of blocks:\newline

Given a target block (encrypted under FHE) denoted $u$,
\begin{enumerate}
\item for each of the $O(\log^2 n)$ blocks $u_i$ on the input level of the circuit, $sim_i := \overline{u}\oplus u_i$,
\item for all $i, j, sim_i[j] := \bigwedge_j sim_i[j],$
\item $out := \sum_i sim_i\cdot data_{u_i}$ (Here, the server sends $out$ to the client); and finally,
\item for all $i$, $u_i := u_i\cdot\overline{sim_i}$ and $data_{u_i} := data_{u_i}\cdot\overline{sim_i}$.
\end{enumerate}

\paragraph{Correctness and Cost Analysis.} 

In the above, after Step (1), we have that for all $i, sim_i = \vec{1}$ \emph{iff} $u_i = u$. Then Step (2) takes the ``internal $AND$'' of all (block-sized) $sim_i$ strings, which gives for all $i,$ if $u_i\ne u$, then $sim_i = \vec{0}$. Therefore, the computation of $out$ in Step (3) is such that $out = data_u$, which is returned to the client as desired. The final step writes a dummy block (a block with $\vec{0}$ as label and data value) over the block whose data was returned to the client via $out$, which completes all computation on one tree of the required $O(\log_c n)$ such computations per ORAM query in the scheme of~\cite{SCSL11}.

Repeating the above procedure across each of the $O(\log_c n)$ trees completes one ORAM read/write query. To bridge the gap between each such query, the client decrypts the FHE layer of $out$, uses the underlying data to prepare the target block $u'$ for the next tree in the sequence, encrypts it under FHE, and sends it to the server. Assuming a sufficiently large block size (with respect to the security parameter $\lambda$) and using standard FHE bit-packing techniques~\cite{BGV11}, there is no direct overhead due to FHE. The only bandwidth overhead comes from the $O(\log_c n)$ repetitions -- once for each of the $O(\log_c n)$ trees in the ORAM. The depth of the circuit is at most $O(\log\log n)$ and has size at most $O(\log^3 n)$, so the server performs at most $poly\log(n)$ computational work per ORAM query sequence.
}

\ignore{
\section{Symmetric-Key Predicate Encryption for RAM Programs}
\elaine{so it only works in the symmetric key setting. need to change
the defn}
\paragraph{Definitions.}
A symmetric-key predicate encryption for RAM programs (\PE-RAM)
scheme is a suite of (possibly randomized) algorithms:
\begin{description}
\item
$(\pp, \sk) \leftarrow \algSetup(1^\lambda)$: a setup algorithm
that outputs public parameters $\pp$ and master secret key $\msk$.
\item
$\sk_f \leftarrow \algKeygen(\msk, f)$: a key generation algorithm
that outputs a key $\sk_f$, 
allowing one to evaluate a {\it RAM program} 
$f$ over encrypted data.
\elaine{in this section, RAM program now means a different thing.
ideally we should unify.}
\item
$C \leftarrow \enc(\msk, D)$: an encryption algorithm which 
encrypts memory array\footnote{We separate 
out the description of the
RAM program $f$ from the array $D$ in this Section for convenience.}
$D$ with the master secret key $\msk$,
and outputs ciphertext $C$.
\item
$y \leftarrow \dec(\sk_f, C)$:
a decryption algorithm that on key $\sk_f$ (corresponding to RAM program 
$f$)
and ciphertext $C$, outputs answer $y$, where 
$y:=f(D)$ in a correct $\PE$-RAM scheme.
\end{description}

%Correctness and security definitions for $\sFE$-RAM are exactly
%the same as the its circuit counterpart as in Section~\ref{sec:prelim}.
\elaine{define correctness and security.}

\paragraph{Construction.}
There are only two key differences between 
VC-RAM and $\PE$-RAM:

First, in VC-RAM, there is conceptually a single encrypted 
data array $D$.
However, in $\PE$-RAM, we can create  
multiple ciphertexts  
for multiple $D$'s, and a secret key 
for a RAM program $f$ should work on any such ciphertext.

Second, in VC-RAM, the server computes an encoded answer 
at the end of the RAM execution, and sends it back for the client
to decode.
In $\PE$-RAM, clear-text output $f(D)$ needs to be revealed at the
end of the RAM execution -- and nothing additional 
should be revealed.

In light of these differences, we can modify our VC-RAM scheme
to achieve a private-index functional encryption scheme for RAM programs
as below:

\begin{description}
\item
$\algSetup(1^\lambda)$:
Let $\sFE$ denote a private-index functional encryption scheme 
for circuits.
Run $(\fmpk, \fmsk) \leftarrow \sFE.\algSetup(1^\lambda)$.

The resulting master secret key $\msk := (\fmpk, \fmsk)$. The public parameters $\pp:=\overline{C}$.
\item
$\algKeygen(\msk, f)$:
%Run $\sFE.\enc(\fmpk, f)$.
garble f as part of memory.
\item
$\enc(\msk, D)$:
garble memory under some nonce tag, and also 
[Run $\overline{C} \leftarrow \sFE.\algKeygen(\fmsk, \algProbgen(nonce, \cdot))$]
\elaine{explain that Probgen is for this specific D, and any f} 
\end{description}

}

\ignore{
\subsection{Secret-key verifiable searchable encryption with access privacy and prover efficiency}

[for query families that can be completed in sublinear time 
with appropriate indexing scheme, our scheme 
achieves sublinear server computation time, i.e., server computation
grows sublinearly in the number of encrypted documents]

\subsection{Authenticated data structures with privacy for arbitrary RAM computation}
}


%\section*{Acknowledgments}
%We thank  Bobby Bhattacharjee, Zvika Brakerski, Dov Gordon,
%   Raluca Ada Popa, Emil Stefanov, and Vinod Vaikuntanathan for useful discussions. 
%We also thank
%Sergey Gorbunov, Vinod Vaikuntanathan, and Hoeteck Wee
%for sharing an early manuscript of~\cite{GVW13} with us.

\bibliographystyle{abbrv}
\bibliography{refs,vc1,vc2,crypto,crypto2}


%\appendix
%\input{appendix}

\end{document}
